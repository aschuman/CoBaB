\subsection{Übersicht}

\subsection{Model}

\subsection{ConfigData}
Die Klasse ConfigData speichert die Benutzereinstellungen Sprache und Benachrichtigungston. Außerdem werden die Hilfe- und Aboutdateien gelesen.

Methoden
\begin{itemize}
\item private ConfigData() Erzeugt ein neues ConfigData Objekt.
\item public ConfigData getInstance() Liefert die ConfigData Instanz zurück. Dies stellt sicher, dass nur eine Instanz von ConfigData existiert.
\item public Language getLanguage() Lädt die vom Benutzer gewählte Sprache.
\item public void setLanguage(Language language) Speichert die vom Benutzer gewählte Sprache.
\item public bool getSoundOn() Lädt die vom Benutzer gewählte Toneinstellung.
\item public void setSoundOn(bool soundOn) Speichert die vom Benutzer gewählte Toneinstellung.
\item public QString getHelp() Lädt die Hilfedatei und gibt deren Inhalt zurück.
\item public QString getAbout() Lädt die Aboutdatei und gibt deren Inhalt zurück.
\end{itemize}

\subsection{<<enumeration>> Language}
Language repräsentiert die verfügbaren Sprachen: Deutsch und Englisch.

\subsection{Annotation}
Eine Annotation ist ein in den Datensätzen vordefinierter Bereich eines Bildes oder Videos.
\begin{itemize}
\item public Annotation(QString id) Gibt eine Instanz dieser Klasse zurück. Die übergebene id identifiziert die Annotation in einem Medium.
\item public QString getId() Gibt die id der Annotation zurück.
\item public AnnotationType getType() Gibt den Typ der Annotation zurück.
\item public friend QDataStream& operator<<(QDataStream& out, Annotation& annotation) Speichert die Annotation in einer Datei.
\item public friend QDataStream& operator>>(QDataStream& in, Annotation& annotation) Lädt die Annotation aus einer Datei.
\end{itemize}

\subsection{View}

\subsection{Controller}