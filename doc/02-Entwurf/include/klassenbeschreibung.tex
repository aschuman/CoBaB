\subsection{Übersicht}

\subsection{Model}
\begin{itemize}

	\item ConfigData \newline
			Diese Klasse liest aus der Config-Textdatei die gewählte Sprache, die Aktivierung bzw. Deaktivierung des 							Benachrichtigungstons und die Texte des Hilfe- und Aboutdialogs ein.
			Werden die Sprache oder die Einstellung des Benachrichtigungstons geändert, überschreibt diese Klasse die 							entsprechenden Werte in der Config-Datei.
	\item SearchResult \newline
			Ein Ergebnis des Suchalgorithmus ist ein Foto oder Video. Das Feedback, das der Benutzer für dieses Foto oder Video 				eingestellt hat, wird zusammen mit dem Ergebnis als SearchResult abgelegt.
	\item SearchResultElement \newline
			Ein Objekt dieser Klasse repräsentiert das gesamte Suchergebnis, es besteht aus einer Liste von SeachResult. Dieses 				Objekt wird als Chronikeintrag und auf Wunsch des Benutzers als Lesezeichen gespeichert.
	\item SearchResultIO \newline
			Diese abstrakte Klasse ermöglicht, neue Suchergebnisse (SearchResultElement) zu speichern und alte Suchergebnisse zu 				laden. Konkret können Lesezeichen mit der Klasse Bookmark und Chronikeinträge mit der Klasse Chronicle verwaltet 					werden.
	\item SearchQuery \newline
			Diese Klasse stellt die Benutzereingaben zusammen, um sie an das Suchverfahren weiterzugeben.
	\item SearchInterface \newline
			Diese Schnittstelle ermöglicht eine einheitliche Verwendung der Suchalgorithmen.
	\item TestAlgorithm \newline
			Dieser Algorithmus implementiert das SearchInterface zu Testzwecken.
	\item DataSet \newline
			Repräsentiert einen Datensatz, es kann sich entweder um einen Foto- oder Videodatensatz handeln.
	\item DataSetInterface \newline
			Diese Schnittstelle bietet der Anwendung und den Suchverfahren einen einheitlichen Zugriff auf die Datensätze.
	\item DataSetComp \newline
			Die Datensatzkompontente implementiert die Datensatzschnittstelle.
	\item DataSetParser \newline
			Parst einen Datensatz.

\end{itemize}

\subsection{View}

\subsection{Controller}