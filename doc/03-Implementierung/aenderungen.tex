Während der Implementierung ergaben sich folgende Änderungen im Entwurf.

\subsection{Modell}
\subsubsection{Annotation}
Annotation bekommt 2 enum-Klassen Type und Form als public Variablen. Über die Methoden getType() und getForm() kann man auf diese Werte zugreifen.

Außerdem wurde die neue Methode copy() hinzugefügt, die eine neue Annotation erstellt und deren Adresse zurückgibt. 

\subsubsection{RectangleAnnotation}
Wie Annotation wird RectangleAnnotation auch eine copy() Methode hinzugefügt.

\subsubsection{Algorithm}
Getter-Methoden wie getName(), getDescription(), getParameter() wurden von SearchAlgorithm und TestAlgorithm in Algorithm verschoben und hier umgesetzt.
Neue Methoden getParameter() und getDefaultParameter() geben die Parameterwerte als QJsonObject (aktuell und standard) zurückt. Die Standardparemeter sind als Metadata gespeichert und werden mittels initialize() gesetzt, wenn der Algorithmus als Plugin geladen wird.

\subsubsection{SearchAlgorithm}
setName() und setDescription() wurden entfernt. Die Matadaten sind nicht zur Laufzeit veränderbar, sondern in einer .json Datei festgespeichert.

\subsubsection{Bookmark}
Ein Bookmark kann in einer Datei gespeichert werden. Die Klasse Bookmark ist daher mit einem default-Konstruktor und einem Konstruktor Bookmark(QString &path) vorgesehen, um das Bookmark-Objekt aus dieser Datei wiederzuherstellen.

Das Bookmark-Objekt bekommt ein neues Attribute : mPath. Den Wert kann man über setPath() setzen. Die Methode saveFile() speichert das Bookmark in die Datei mit diesem Pfad, und deleteFile() löscht die Datei, falls vorhanden.

Um 2 Bookmarks nach Datum oder Name zu vergleichen, sind die Methoden smallerByName() und smallerByDate() hinzugefügt.

validate(Bookmark &b) prüft ein Bookmark-Objekt auf die Gültigkeit, wie z.Bsp. ob der Datensatz noch vorhanden ist.

\subsubsection{BookmarkList}
Die Methode save(QString &path) wurde entfernt, weil nun nicht alle Bookmarks in der Liste zusammengespeichert werden, sondern jede in eine Datei. Wenn ein Bookmark hinzugefügt bzw. aus der Liste entfernt ist, wird die entsprechende Datei erstellt bzw. gelöscht.

\subsection{View}
\subsubsection{ViewerPageWidget}
Da sich der Photo- und der SingleFrameVideoViewer in ihren UIs nur um ein paar Buttons unterscheiden, nutzen sie die gemeinsame UI des ViewerPageWidget. Der PhotoViewer blendet die nicht benötigten UI Elemente aus. Daher sind die Buttons des SingleFrameVideoViewers in die Basisklasse verschoben worden.

Der Entwurf war für das GraphicsSingleFrameVideoItem so nicht umsetzbar, da es keinen Sinn macht, dass das GraphicsSingleFrameVideoItem, das im CustomGraphicsView als einzelnes Frame angezeigt wird, einen Player besitzt. Daher wurde die Abspiellogik direkt im SingleFrameVideoViewer implementiert und die beiden Klassen (GraphicsSingleFrameVideoItem und SingleFrameVideoPlayer) wurden weggelassen. Für die Videoframes wurden stattdessen ebenfalls ClickableGraphicsPixmapItems verwendet, was zudem näher an der Implementierung des PhotoViewers liegt.

Das CustomGraphicsView sollte erst ein QGraphicsView besitzen, erbt aber nun direkt von QGraphicsView, um das Zoomen zu ermöglichen. Da es nun ein QGraphicsView ist, wurden die Buttons ins ViewerPageWidget verschoben.

In den ClickableGraphicsItems wurde außerdem nicht das mousePressEvent, sondern das contextMenuEvent überschrieben, da nur der Rechtsklick abgefangen werden sollte, um ein Kontextmenü anzuzeigen.

Der VideoViewer und das zugehörige ClickableQGraphicsVideoItem wurden aus Zeitgründen nicht implementiert.

\subsection{Controller}
Die Klassen BookmarkManager und MainWindowManager sowie SearchResultManager wurden nicht umgesetzt.

\subsubsection{SearchManager}
\inlinecode{SearchManager(resWind : ResultsWindow*)} wird zu \inlinecode{SearchManager(resultsPageWidget : ResultsPageWidget*)}. Hierbei handelt es sich um eine versäumte Anpassung, die bereits in der Entwurfsphase vorgenommen hätte werden müssen.

\subsubsection{AlgorithmThread}
Die AlgorithmThread-Klasse war im Entwurf nicht vorgesehen. Sie wurde nötig, um die Ergebnisse eines vollendeten Suchvorgangs einsammeln zu können ohne das Algorithmeninterface zu ändern. AlgorithmThread speichert die Ergebnisse bis sie von SearchManager abgefragt werden.

\subsubsection{PageStackFrame}
Die Methode void \inlinecode{incrementSize()} wurde hinzugefügt. Sie dient lediglich der Bequemlichkeit.

\subsubsection{PageType}
Die Elemente \inlinecode{PHOTO\_VIEWER}, \inlinecode{SINGLE\_FRAME\_VIDEO\_VIEWER} sowie \inlinecode{VIDEO\_VIEWER} sind hinzugekommen. \inlinecode{VIEWER} ist weggefallen. Hierbei handelt es sich um eine versäumte Anpassung, die bereits in der Entwurfsphase vorgenommen hätte werden müssen. Der Entwurf des View sieht vor, dass PhotoViewer, SingleFrameVideoViewer und VideoViewer PageWidgets sind und als solche beim Navigator registriert werden. Dies ernötigt die zusätzlichen PageTypes. \inlinecode{PageType::Viewer} erübrigt sich dadurch.

\subsubsection{PageRegistration}
PageRegistration bekommt einen Konstruktor \inlinecode{PageRegistration(widget : PageWidget)}, der im Entwurf fehlt. Zusätzlich ist ein explizit deklarierter und manuell definierter Move-Konstruktor und Move-Zuweisungsoperator hinzugefügt worden. Der Rückgabewert von get Widget ist PageWidget, was im Entwurf nicht eingetragen wurde.

\subsubsection{Navigator}
Alle Container Typen aus Qt wurden im Navigator durch ebenbürtige Typen C++-Standardbibliothek ersetzt. QVariant ist von dieser Änderung ausgenommen. Grund sind die oft strengeren Bedingungen der Qt-Typen an die benutzerdefinierten Typen. Beispielsweise fordert \inlinecode{QList<T>} einen Standardkonstruktor in T.
Die Slots \inlinecode{toPreviousPage()} und \inlinecode{toHomePage()} wurden für Signale des MainWindow hinzugefügt und waren im Entwurf versäumt.
Der Slot \inlinecode{tryRead(index : size\_t, value : QVariant)} wurde zu \inlinecode{tryRead(index : int, value : QVariant)}. Diese Änderung vereinfacht die Benutzung des Stacks des Navigators (nachzulesen in der Sourcecodedokumentation).
