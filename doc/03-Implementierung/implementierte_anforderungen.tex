% !TeX spellcheck = de_DE_frami

\newcommand\chk{\marginpar{$\CheckedBox$}}
\newcommand\nochk{\marginpar{$\Box$}}

\subsection{Pflichtanforderungen}
\begin{enumerate} [label=\bfseries /F \arabic*0/, leftmargin=*]
	\item Starten des Programms \chk
	\item Beenden des Programms \chk
	\item Automatisches Erkennen verfügbarer Suchverfahren \chk
	\item Übergeben eines Standardordners für die Bibliotheksdatensätze über die Kommandozeile \chk
	\item Anzeigen eines Hilfe-Dialogs mit Hinweisen zur Benutzung des Programms \chk
	\item Anzeigen eines About-Dialogs mit Informationen zum Programm \chk
	\item Rückkehr zur Bibliothek \chk
	\newline
 
	\item Anzeigen einer Bibliothek von Datensätzen \chk
	\item Auswählen eines Datensatzes, der nicht in der Bibliothek enthalten ist \chk
	\item Auswählen eines Datensatzes aus der Bibliothek \chk
	\item Anzeigen einer Übersicht der Bilder bzw. Videos des gewählten Datensatzes \chk
	\item Unterstützung der Bildformate JPEG, PNG und BMP \chk
	\item Anzeigen einer größeren Darstellung eines ausgewählten Bildes bzw. Videos des Datensatzes \chk
	\item Abspielen eines Videos aus Einzelbildern \chk
	\item Auswählen des vorherigen oder nächsten Bildes bzw. Videos für die große Ansicht \chk
	\item Auswählen eines Bildes bzw. Videos aus dem gewählten Datensatz als Suchvorlage \chk
	\item Anzeigen annotierter Bildbereiche \chk
	\item Auswahl eines annotierten Bildbereiches als Suchvorlage \chk
	\item Einschränkung der Suchvorlage auf einen benutzerdefinierten rechteckigen Bereich (falls keine Annotation gewählt wurde) \chk
	\item Auswahl eines für die gewählte Suchvorlage geeigneten Suchverfahrens \chk
	\item Anzeigen einer Beschreibung für die Suchverfahren \chk
	\item Festlegen der für das gewählte Suchverfahren spezifischen Parameter \chk
	\item Anzeigen sämtlicher vorgenommener Einstellungen zur aktuellen Suche, um eine Überprüfung durch den Benutzer zu ermöglichen \chk
	\item Ändern beliebiger zu einer Suche vorgenommener Einstellungen \chk
	\newline
	\item Starten der Suche \chk
	\item Anzeigen einer Fortschrittsanimation während des Suchvorgangs \chk
	\newline
	\item Anzeigen einer sortierten Übersicht der Suchergebnisse \chk
	\item Optionales Einstellen eines Feedbacks zu einem Suchergebnis und Starten einer weiteren Suche mit diesem Feedback und dem selben Suchverfahren \chk
	
	\item Loggen von Informationen, Warnungen, Fehlern und Debugmeldungen \chk
	 
\end{enumerate}

\subsection{Wunschanforderungen}
\begin{enumerate} [label=\bfseries /FW \arabic*0/, leftmargin=*]
	\item Nicht-flüchtige Auswahl der Übersetzungen der GUI in Deutsch und Englisch \chk
	\newline
	\item Wählen von mehreren Datensätzen, in denen gesucht wird \chk
	\item Generierung eines zufälligen oder festgelegten Vorschaubildes für den Datensatz \chk
	\newline
	\item Abspielen eines Benachrichtigungstons bei Abschluss einer Suche \chk
	\item Ein-/Abschalten des Benachrichtigungstons \chk
	\item Abspielen von Videos aus Videodateien (mit Ausgabe der Audiospur)\nochk
	\newline
	\item Festlegung eines Wertes zwischen 0 und 10 als Feedback\nochk
	\item Durchführen einer weiteren Suche in den Ergebnissen einer bereits beendeten Suche mit anderem Suchverfahren\nochk
	\newline
	\item Wechsel in einen Vollbildmodus für Präsentation \chk
	\item Zoomen und Scrollen in der größeren Darstellung im Fotoanzeiger oder Videoplayer \chk
	\item Anzeigen einer größeren Darstellung eines ausgewählten Suchergebnisses\nochk
	\newline
	\item Nicht-flüchtiges Speichern der zuletzt verwendeten Datensätze \chk
	\item Nicht-flüchtiges Speichern der Lesezeichen\nochk
	\item Automatisches nicht-flüchtiges Speichern der Suchchronik\nochk
	\item Anzeigen der Historie der Datensätze \chk
	\item Anzeigen einer Übersicht der Lesezeichen\nochk
	\item Anzeigen einer Übersicht der Suchchronik\nochk
	\item Wiederanzeigen der Ergebnisse eines Lesezeichens\nochk
	\item Wiederanzeigen der Ergebnisse der Suchchronik\nochk
	\newline
	\item Abbrechen der Suche mit Weiterführung der Oberfläche\nochk
\end{enumerate}
\pagebreak
