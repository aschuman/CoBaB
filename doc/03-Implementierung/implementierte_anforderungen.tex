% !TeX spellcheck = de_DE_frami
\subsection{Pflichtanforderungen}
\begin{enumerate} [label=\bfseries /F \arabic*0/, leftmargin=*]
	\item DONE Starten des Programms
	\item DONE Beenden des Programms
	\item DONE Automatisches Erkennen verfügbarer Suchverfahren
	\item DONE Übergeben eines Standardordners für die Bibliotheksdatensätze über die Kommandozeile
	\item DONE Anzeigen eines Hilfe-Dialogs mit Hinweisen zur Benutzung des Programms
	\item DONE Anzeigen eines About-Dialogs mit Informationen zum Programm
	\item DONE Rückkehr zur Bibliothek
	\newline
 
	\item DONE Anzeigen einer Bibliothek von Datensätzen
	\item DONE Auswählen eines Datensatzes, der nicht in der Bibliothek enthalten ist
	\item DONE Auswählen eines Datensatzes aus der Bibliothek
	\item DONE Anzeigen einer Übersicht der Bilder bzw. Videos des gewählten Datensatzes
	\item DONE Unterstützung der Bildformate JPEG, PNG und BMP
	\item DONE Anzeigen einer größeren Darstellung eines ausgewählten Bildes bzw. Videos des Datensatzes
	\item DONE Abspielen eines Videos aus Einzelbildern
	\item DONE Auswählen des vorherigen oder nächsten Bildes bzw. Videos für die große Ansicht
	\item DONE Auswählen eines Bildes bzw. Videos aus dem gewählten Datensatz als Suchvorlage
	\item DONE Anzeigen annotierter Bildbereiche
	\item DONE Auswahl eines annotierten Bildbereiches als Suchvorlage
	\item DONE Einschränkung der Suchvorlage auf einen benutzerdefinierten rechteckigen Bereich (falls keine Annotation gewählt wurde)
	\item DONE Auswahl eines für die gewählte Suchvorlage geeigneten Suchverfahrens
	\item DONE Anzeigen einer Beschreibung für die Suchverfahren
	\item DONE Festlegen der für das gewählte Suchverfahren spezifischen Parameter
	\item DONE Anzeigen sämtlicher vorgenommener Einstellungen zur aktuellen Suche, um eine Überprüfung durch den Benutzer zu ermöglichen
	\item DONE Ändern beliebiger zu einer Suche vorgenommener Einstellungen
	\newline
	\item DONE Starten der Suche
	\item DONE Anzeigen einer Fortschrittsanimation während des Suchvorgangs
	\newline
	\item DONE Anzeigen einer sortierten Übersicht der Suchergebnisse
	\item DONE Optionales Einstellen eines Feedbacks zu einem Suchergebnis und Starten einer weiteren Suche mit diesem Feedback und dem selben Suchverfahren
	
	\item DONE Loggen von Informationen, Warnungen, Fehlern und Debugmeldungen
	 
\end{enumerate}

\subsection{Wunschanforderungen}
\begin{enumerate} [label=\bfseries /FW \arabic*0/, leftmargin=*]
	\item DONE Nicht-flüchtige Auswahl der Übersetzungen der GUI in Deutsch und Englisch
	\newline
	\item DONE Wählen von mehreren Datensätzen, in denen gesucht wird
	\item DONE Generierung eines zufälligen oder festgelegten Vorschaubildes für den Datensatz
	\newline
	\item DONE Abspielen eines Benachrichtigungstons bei Abschluss einer Suche
	\item DONE Ein-/Abschalten des Benachrichtigungstons
	\item Abspielen von Videos aus Videodateien (mit Ausgabe der Audiospur)
	\newline
	\item Festlegung eines Wertes zwischen 0 und 10 als Feedback
	\item Durchführen einer weiteren Suche in den Ergebnissen einer bereits beendeten Suche mit anderem Suchverfahren
	\newline
	\item DONE Wechsel in einen Vollbildmodus für Präsentation
	\item DONE Zoomen und Scrollen in der größeren Darstellung im Fotoanzeiger oder Videoplayer
	\item Anzeigen einer größeren Darstellung eines ausgewählten Suchergebnisses
	\newline
	\item DONE Nicht-flüchtiges Speichern der zuletzt verwendeten Datensätze
	\item Nicht-flüchtiges Speichern der Lesezeichen
	\item Automatisches nicht-flüchtiges Speichern der Suchchronik
	\item DONE Anzeigen der Historie der Datensätze
	\item Anzeigen einer Übersicht der Lesezeichen
	\item Anzeigen einer Übersicht der Suchchronik
	\item Wiederanzeigen der Ergebnisse eines Lesezeichens
	\item Wiederanzeigen der Ergebnisse der Suchchronik
	\newline
	\item Abbrechen der Suche mit Weiterführung der Oberfläche
\end{enumerate}
\pagebreak
