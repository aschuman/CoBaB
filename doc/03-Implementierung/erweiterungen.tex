Dieser Abschnitt beschäftigt sich mit Erweiterungen, die während des Entwurfs und der Implementierung der Erstversion von CoBaB bedacht wurden.

\subsection{PageWidgets}
Die Benutzeroberfläche von CoBaB ist in sogenannte PageWidgets, von denen zu jeder Zeit genau eines angezeigt wird, zerteilt.
Daten, auf die von mehreren PageWidgets zugegriffen wird, werden über einen vom Navigator verwalteten Stack ausgetauscht.
Dadurch sind die PageWidgets im objektorientierten Sinne voneinander unabhängig, es können also Änderungen an einem PageWidget vorgenommen werden ohne dabei ein anderes zu betreffen.
Bei Änderungen ist jedoch zu beachten, dass ein aktives PageWidget bestimmte Daten in einer bestimmten Reihenfolge im Stack erwartet.
Unbedachte Änderungen an den Operationen, die den Stack betreffen, können die Funktionalität anderer PageWidgets beeinträchtigen.

Um ein neues PageWidget hinzuzufügen, muss zunächst ein neues Element in der Aufzählung PageType angelegt werden. Unter dieser wird ein neues PageWidget-Objekt in MainControl beim Navigator registriert. Um es anzuzeigen, muss es entweder beim Start des Navigators als initiales PageWidget angegeben werden oder es muss beim Navigator eine Transition, die das neue PageWidget als Ziel hat, registriert werden, damit es im Laufe des Programms eingeblendet wird. Auch hier ist zu beachten, dass PageWidgets, die auf das neue PageWidget folgen, den Stack in einem bestimmten Zustand erwarten.

\subsection{AnnotationDrawer}
Um Annotationen anderer Formen (z.B. Kreise, Ellipsen) anzeigen zu können, muss die neue Annotation von Annotation erben. Außerdem muss ein entsprechendes ClickableGraphicsItem erstellt werden, das von QGraphicsItem (z.B. QGraphicsEllipseItem) erbt. Um anklickbar zu sein, muss das Item das contextMenuEvent überschreiben und ein Signal mit der gewählten Annotation und der Mausposition aussenden. In Annotation muss die neue Form hinzugefügt werden und die fromStream Methode angepasst werden, um die richtige Annotation zu lesen. In der AnnotationGraphicsItemFactory muss, abhängig von der Form, das neue GraphicsItem erzeugt werden.

\subsection{Suchalgorithmen und Plugins}
Hinzufügen von Suchverfahren ist ein wesentliches Kriterium in der Entwicklung von CoBaB, weshalb Suchalgorithmen als Plugins implementiert werden. Qt bietet eine eigene API für das Umsetzen von Plugins und Plugin-Schnittstelle (man beachte hierzu auch die Qt-Dokumentation für Low-Level-Plugins). Ein neues CoBaB-Plugin umfasst ein Qt-Projekt, das neben den für Qt-Plugins notwendigen Einstellungen auch das CoBaB-Unterprojekt \enquote{interface} bindet. Dieses enthält eine Klasse, die das Plugin-Interface implementiert. Das Plugin-Interface wird duch die Klasse \enquote{Algorithm} angeboten. Es ist auch möglich eine der Unterklassen von \enquote{Algorithm}, die zusätzliche FUnktionen bieten, zu erben. (Man beachte die Dokumentation von \enquote{Algorithm} und den Unterklassen).
Auch Plugins die keine Suchalgorithmen implementieren sind denkbar und wurden im Entwurf bedacht. Sie verfügbar zu machen bedarf im Regelfall einer Anpassung in CoBaB selbst, jedoch keine Anpassung der Plugin-Schnittstelle.
