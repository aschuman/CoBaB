\subsection{Projektstruktur}
Um Kompenententests nicht mit dem echten Produktcode zu vermengen wurde ein eigenes Qt-Projekt \enquote{test.pro}
angelegt, das nur Tests enthält und eine eigene ausführbare Datei erzeugt. Desweiteren wurde der Produktcode in zwei weitere Qt-Projekte unterteilt. Die \enquote{app.pro} enthält nur eine Quellcodedatei \enquote{main.cpp} und erzeugt als ausführbare Datei das fertige Produkt. Der verbleibende Quellcode stellt den Kern des Projekts dar und wird von \enquote{core.pro} umfasst. Erzeugt wird eine Bibliothek. Hierdurch wird ermöglicht den Kern statisch in die ausführbare Datei des Testprojekts zu binden. Auch in das ausführbare Produkt ist der Kern statisch gebunden.

\subsection{Pointer Ownership}
Durch die verwendung von C++14 (siehe Abschnitt 11.1 des Pflichtenhefts) ist die verwendung von unique\_ptr, die selbständig ihre Ziele löschen, möglich. Ebenso übernehmen einige Komponenten der Qt-Bibliothek den Besitz übergebene Zeiger. Dynamisch allokierte Objekte, die gleichzeitig von einem unique\_ptr und einer der besagen Qt-Komponenten besessen werden, könnten zwei Versuche das selbe Objekt zu löschen zur Folge haben. Dies ist undefiniertes verhalten. Es müssen beim Verwenden von unique\_ptr also mehrfache Löschungen beachtet werden.

\subsection{Logger}
Der verwendete Logger ist aus dem Artikel \enquote{A Lightweight Logger for C++} von Filip Janiszewski entnommen.
Dieser ist zu finden unter http://www.drdobbs.com/cpp/a-lightweight-logger-for-c/240147505.
Er verwendet keine externen Bibliotheken und ist Threadsicher, weshalb die wahl auf ihn gefallen ist.

\subsection{Move-Konstruktoren}
Die Verwendung von unique\_ptr in Komponenten, die beispielsweise durch Haltung von Instanzen in einem Container die Möglichkeit haben müssen verschoben (move) zu werden, macht die Implementierung eines Move-Konstruktors und eines Move-Zuweisungsoperators notwendig. In C++14 (im Speziellen C++11) ist die automatische Generierung solcher Funktionen vorgesehen, falls dies durch attributweise Durchführung möglich ist. Die auf der Plattform Windows verwendete Version 2013 des Visual Studio Compilers unterstützt dies jedoch nicht. (siehe https://msdn.microsoft.com/en-us/library/dn457344.aspx)
Komponenten die notwendigerweise Move unterstützen, enthalten daher manuell geschriebene Move-Konstruktoren und -Zuweisungsoperatoren. (zum Beispiel: PageRegistration)
