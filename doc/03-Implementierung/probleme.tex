Dieser Abschnitt beschäftigt sich mit wesentlichen Problemen während der 
Implementierung und in diesem Projekt gewählten Lösungen bzw. Übergangslösungen.
Probleme, die den Entwurf betreffen, werden in einem eigenen Abschnitt beleuchtet.

\subsection{Qt und \texttt{unique\_ptr}}
Durch die verwendung von C++14 (siehe Abschnitt 11.1 des Pflichtenhefts) ist die verwendung von unique\_ptr, die selbständig ihre Ziele löschen, möglich. Ebenso übernehmen einige Komponenten der Qt-Bibliothek den Besitz übergebene Zeiger. Dynamisch allokierte Objekte, die gleichzeitig von einem unique\_ptr und einer der besagen Qt-Komponenten besessen werden, könnten zwei Versuche das selbe Objekt zu löschen zur Folge haben. Dies ist undefiniertes verhalten. Es müssen beim Verwenden von unique\_ptr also mehrfache Löschungen beachtet werden.

\subsection{Move-Konstruktoren mit dem Visual Studio 2013 Compiler}
Die Verwendung von unique\_ptr in Komponenten, die beispielsweise durch Haltung von Instanzen in einem Container die Möglichkeit haben müssen verschoben (move) zu werden, macht die Implementierung eines Move-Konstruktors und eines Move-Zuweisungsoperators notwendig. In C++14 (im Speziellen C++11) ist die automatische Generierung solcher Funktionen vorgesehen, falls dies durch attributweise Durchführung möglich ist. Die auf der Plattform Windows verwendete Version 2013 des Visual Studio Compilers unterstützt dies jedoch nicht. (siehe https://msdn.microsoft.com/en-us/library/dn457344.aspx)
Komponenten die notwendigerweise Move unterstützen, enthalten daher manuell geschriebene Move-Konstruktoren und -Zuweisungsoperatoren. (zum Beispiel: PageRegistration)

\subsection{Qt-Plugins und statische Bibliotheken}
Die verwendete Projektstruktur hat es nötig gemacht, dass sowohl das Unterprojekt \enquote{app.pro} als auch die zur Laufzeit dynamisch gebundenen Plugins das Unterprojekt \enquote{core.pro} verwenden.
Die Annahme die Plugins bräuchten \enquote{core.pro} nicht binden, da sie ja vom Programm, das \enquote{core.pro} bindet, gebunden wird, stellte sich als falsch heraus.
Die verwendete Übergangslösung \enquote{core.pro} statisch in die Plugins zu binden, birgt den Nachteil, dass die Plugins unnötig groß werden.
