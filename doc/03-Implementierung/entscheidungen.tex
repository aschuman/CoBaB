Dieser Abschnitt nennt und beschreibt wesentliche Entscheidungen, die während der Implementierung vorgenommen wurden. Dies betrifft nicht Entscheidungen wie das Lösen von konkreten, vorgefallenen Problemen oder das Ändern des Enturfes, die in gesonderten Abschnitten behandelt werden.

\subsection{Projektstruktur}
Um Kompenententests nicht mit dem echten Produktcode zu vermengen wurde ein eigenes Qt-Projekt \enquote{test.pro}
angelegt, das nur Tests enthält und eine eigene ausführbare Datei erzeugt. Desweiteren wurde der Produktcode in zwei weitere Qt-Projekte unterteilt. Die \enquote{app.pro} enthält nur eine Quellcodedatei \enquote{main.cpp} und erzeugt als ausführbare Datei das fertige Produkt. Der verbleibende Quellcode stellt den Kern des Projekts dar und wird von \enquote{core.pro} umfasst. Erzeugt wird eine Bibliothek. Hierdurch wird ermöglicht den Kern statisch in die ausführbare Datei des Testprojekts zu binden. Auch in das ausführbare Produkt ist der Kern statisch gebunden.

\subsection{Logger}
Der verwendete Logger ist dem Artikel \enquote{A Lightweight Logger for C++} von Filip Janiszewski entnommen.
Dieser ist zu finden unter http://www.drdobbs.com/cpp/a-lightweight-logger-for-c/240147505.
Er verwendet keine externen Bibliotheken und ist Threadsicher, weshalb die Wahl auf ihn fiel.
