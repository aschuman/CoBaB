\newglossaryentry{Annotation}
{
name=Annotation,
description={ist ein in den Datensätzen vordefinierter Bereich eines Bildes oder Videos, z.B. Person oder Gesicht}
}

\newglossaryentry{Lesezeichen}
{
name=Lesezeichen,
description={ist ein Link für schnelleren Zugriff auf bestimmte, meist häufig besuchte Ergebnisse in einer Lesezeichen-Sammlung}
}

\newglossaryentry{Suchchronik}
{
name=Suchchronik,
description={ist eine Sammlung von gespeicherten zuletzt generierten Suchergebnissen im Programm}
}

\newglossaryentry{Feedback}
{
name=Feedback,
description={umfasst das Bewerten der generierten Ergebnisse und sein Weiterleiten an die \glslink{Suchverfahren}{Suchalgorithmen} für eine verbesserte Suche}
}

\newglossaryentry{GUI}
{
name=GUI,
description={(engl. graphical user interface) Grafische Benutzeroberfläche}
}

\newglossaryentry{Dialog}
{
name=Dialog,
description={ist ein Element der grafischen Benutzeroberfläche, bei dem Eingaben vom Benutzer eingeholt werden}
}

\newglossaryentry{Versionsverwaltung}
{
name=Versionsverwaltung,
description={Versionsverwaltung ist ein System, dass Änderungen an Dateien erfasst und mit einem Zeitstempel dokumentiert, um beliebige vorherige Versionen wiederherzustellen. Der Gebrauch von Versionsverwaltung heißt Versionierung}
}

\newglossaryentry{Git}
{
name=Git,
description={ist eine freie Software zur \gls{Versionsverwaltung} von Dateien}
}

\newglossaryentry{Phase}
{
name=Phase,
description={ist einer der Abschnitte im Verlauf der Entwicklung eines Softwareprodukts: Planung, Entwurf, Implementierung, Testen, Wartung und Pflege}
}

\newglossaryentry{Phasendokument}
{
name=Phasendokument,
description={ist das Resultat einer \gls{Phase} der Softwareentwicklung}
}

\newglossaryentry{Qt}
{
name=Qt,
description={ist eine Klassenbibliothek, also eine Sammlung von Routinen und Hilfsmitteln in Form von Programmcode, die die Darstellung von grafischen Benutzeroberflächen erlaubt. Sie ist in der Programmiersprache C++ verfasst}
}

\newglossaryentry{Qt Creator}
{
name=Qt Creator,
description={ist eine integrierte Entwicklungsumgebung - ein Hilfsmittel bei der Erstellung von Programmcode}
}

\newglossaryentry{Qt Designer}
{
name=Qt Designer,
description={ist ein Hilfsmittel zum Entwerfen und Erstellen grafischer Benutzeroberflächen}
}

\newglossaryentry{Qt Test}
{
name=Qt Test,
description={ist ein Framework für Modultests - ein Hilfsmittel beim kontinuierlichen Testen eines Programms bereits während der Implementierung}
}

\newglossaryentry{Suchverfahren}
{
name=Suchverfahren,
description={ist eine Routine, die die Bilder bzw. Videos eines gegebenen Datensatzes in ihrer Ähnlichkeit mit einer gegebenen Suchvorlage bewertet. \newline Wird als Synonym von Suchalgorithmen im Dokument verwendet}
}

\newglossaryentry{Widget}
{
name=Widget,
description={ist eine verallgemeinernde Bezeichnung für Komponenten einer \gls{GUI}}
}

\newglossaryentry{Entwurfsmuster}
{
name=Entwurfsmuster,
description={ist eine Lösung für häufig auftretende Probleme beim Entwurf von Software}
}

\newglossaryentry{Model-View-Controller}
{
name=Model-View-Controller,
description={ist ein Muster zur Strukturierung von Software-Entwicklung in die drei Einheiten Datenmodell (engl. model), Präsentation (engl. view) und Programmsteuerung (engl. controller). Ziel des Musters ist ein flexibler Programmentwurf, der eine spätere Änderung oder Erweiterung erleichtert und eine Wiederverwendbarkeit der einzelnen Komponenten ermöglicht}
}

\newglossaryentry{Doxygen}
{
name=Doxygen,
description={ist eine freie Software, die aus Quellcode und darin enthaltenen Kommentaren eine Softwaredokumentation erzeugt}
}
