% !TeX spellcheck = de_DE_frami
\subsection{Pflichtanforderungen}
\begin{enumerate} [label=\bfseries /F \arabic*0/, leftmargin=*]
	\item Starten des Programms \label{f:programmstart}
	\item Beenden des Programms \label{f:beenden}
	\item Automatisches Erkennen verfügbarer Suchverfahren \label{f:erkennen_suchverfahren}
	\item Anzeigen eines Hilfe-\glslink{Dialog}{Dialogs} mit Hinweisen zur Benutzung des Programms \label{f:hilfe}
	\item Anzeigen eines About-\glslink{Dialog}{Dialogs} mit Informationen zum Programm \label{f:about}
	\newline
%	\item[\req{F 0}] Er\"offnen eines neuen Suchkontexts
%	\item[\req{F 0}] Er\"offnen eines neuen \glslink{Suchkontext}{Suchkontexts}
 
	\item Anzeigen einer Bibliothek von Datensätzen, wie in Abbildung \ref{fig:bibliothek} zu sehen ist \label{f:bibliothek_anzeigen}
	\item Hinzufügen von Datensätzen zur Bibliothek, wie in Abbildung \ref{fig:filedialog} zu sehen ist \label{f:datensatz_hinzufuegen}
	\item Ausw\"ahlen eines Datensatzes aus der Bibliothek \label{f:auswahl_eines_datensatzes}
	\item Anzeigen einer Übersicht der Bilder bzw. Videos des gewählten Datensatzes, wie in Abbildung \ref{fig:fotoanzeiger} zu sehen ist \label{f:uebersicht_anzeigen}
	\item Unterstützung der Bildformate JPEG, PNG und BMP
	\item Anzeigen einer größeren Darstellung eines ausgewählten Bildes bzw. Videos des Datensatzes \label{f:anzeigen_groessere_darstellung}
	\item Abspielen eines Videos aus Einzelbildern \label{f:video_abspielen}
	\item Auswählen des vorherigen oder nächsten Bildes bzw. Videos für die große Ansicht \label{f:vorheriges_naechstes}
	\item Ausw\"ahlen eines Bildes bzw. Videos aus dem gewählten Datensatz als Suchvorlage \label{f:bildauswahl}
	\item Anzeigen \glslink{Annotation}{annotierter Bildbereiche} \label{f:annotation_anzeigen}
	\item Auswahl eines \glslink{Annotation}{annotierten Bildbereiches} als Suchvorlage \label{f:annotation_auswaehlen}
	\item Einschränkung der Suchvorlage auf einen benutzerdefinierten rechteckigen Bereich (falls keine \gls{Annotation} gewählt wurde) \label{f:bereich_auswaehlen}
	\item Auswahl eines für ggf. gewählte \gls{Annotation} geeigneten Suchverfahrens \label{f:auswahl_suchverfahren}
	\item Anzeigen einer Beschreibung für die Suchverfahren \label{f:beschreibung_suchverfahren}
	\item Festlegen der für das gewählte Suchverfahren spezifischen Parameter, wie in Abbildung \ref{fig:parameterauswahl} zu sehen ist \label{f:parameterwahl}
	\item Anzeigen sämtlicher vorgenommener Einstellungen zur aktuellen Suche, um Überprüfung durch den Benutzer zu ermöglichen, wie in Abbildung \ref{fig:überprüfung} zu sehen ist \label{f:ueberpruefung}
	\item Ändern sämtlicher vorgenommener Einstellungen nach der Überprüfung \label{f:einstellungen_nach_ueberpruefung_aendern}
	\newline
	\item Starten der Suche \label{f:suche_starten}
	\item Anzeigen einer Fortschrittsanimation während des Suchvorgangs \label{f:fortschrittsanimation}
	\newline
	\item Anzeigen einer \"Ubersicht der Suchergebnisse, wie in Abbildung \ref{fig:suchergebnisse} \label{f:suchergebnisse_anzeigen}
	\item Einstellen eines Feedbacks zu einem Suchergebnis und Starten einer weiteren Suche mit diesem Feedback und dem selben Suchverfahren \label{f:feedback}
	
	\item Loggen von Informationen, Warnungen, Fehlern und Debugmeldungen
	 
\end{enumerate}

\subsection{Wunschanforderungen}
\begin{enumerate} [label=\bfseries /FW \arabic*0/, leftmargin=*]
	\item nicht-flüchtige Auswahl der Übersetzungen der \gls{GUI} in Deutsch und Englisch \label{fw:sprache}
	\newline
	\item Wählen einer Suchvorlage, die nicht in den gewählten Datensätzen enthalten ist (erweitert \ref{f:bildauswahl})
	\item Wählen von mehreren Datensätzen in denen gesucht wird (erweitert \ref{f:auswahl_eines_datensatzes})
	\newline
	\item Abspielen eines Benachrichtigungstons bei Abschluss einer Suche \label{fw:signalton}
	\item Ein-/Abschalten des Benachrichtigungstons \label{fw:signalton_einaus}
	\item Abspielen von Videos aus Videodateien (mit Ausgabe der Audiospur) \label{fw:echte_videos_abspielen}
	\item Durchführen einer weiteren Suche in den Ergebnissen einer bereits beendeten Suche mit anderem Suchverfahren \ref{f:feedback}
	\newline
	\item Wechsel in einen Vollbildmodus für Präsentation
	
	\item Zoomen und Scrollen in der größeren Darstellung in der Datensatzauswahl
	\item Generierung eines zufälligen oder festgelegten Vorschaubildes für den Datensatz
	\item Anzeigen einer größeren Darstellung eines ausgewählten Suchergebnisses \label{fw:groesseres_suchergebnis}
	
	\item nicht-flüchtiges Speichern der Lesezeichen (siehe auch \ref{fw:lesezeichen}) \label{fw:speichern_lesezeichen}
	\item automatisches nicht-flüchtiges Speichern der Chronik (siehe auch \ref{fw:chronik}) \label{fw:speichern_chronik}
	\item Anzeigen einer Übersicht der Lesezeichen \label{fw:lesezeichen_uebersicht}
	\item Anzeigen einer Übersicht der Chronik \label{fw:chronik_uebersicht}
	\item Wiederanzeigen der Ergebnisse eines Lesezeichens (siehe auch \ref{fw:speichern_lesezeichen}) \label{fw:lesezeichen}
	\item Wiederanzeigen der Ergebnisse der Chronik (siehe auch \ref{fw:speichern_chronik}) \label{fw:chronik}
	
	\item Abbrechen der Suche mit Weiterführung der Oberfläche
	\item Festlegung eines Wertes zwischen 0 und 10 als Feedback
\end{enumerate}
\pagebreak
