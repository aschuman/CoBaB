% !TeX spellcheck = de_DE_frami
\begin{enumerate} [label=\bfseries /F \arabic*0/]
	\subsection{Pflicht}
	\item Starten des Programms
	\item Beenden des Programms
	\item Erkennen verfügbarer Datensätze und Suchverfahren
	\item Anzeigen eines Help-Menüs
	\item Anzeigen eines About-Menüs
	\item Auswahl der Sprache
	\newline
%	\item Er\"offnen eines neuen Suchkontexts
%	\item Er\"offnen eines neuen \glslink{Suchkontext}{Suchkontexts}
	\item Ausw\"ahlen der Datens\"atze
	\item Anzeigen einer Übersicht der Bilder bzw. Videos der gewählten Datensätze
	\item Ausw\"ahlen eines Bildes bzw. Videos aus den gewählten Datensätzen als Suchvorlage
	\item Auswahl eines annotierten Bildbereiches der Vorlage
	\item Einschränkung der Suchvorlage auf einen benutzerdefinierten rechteckigen Bereich (falls keine Annotation gewählt wurde)
	\item Auswahl des Suchverfahrens
	\newline
	\item Starten der Suche
	\item Anzeigen einer Sanduhr während des Suchvorgangs
	\item Abbrechen der Suche
	\newline
	\item Anzeigen einer \"Ubersicht der Suchergebnisse
	\item Auswahl und Anzeige eines einzelnen Suchergebnisses
	\item Abspielen eines Videos aus Einzelbildern
	\item Einstellen eines Feedbacks zu einem Schergebnis
	\item \"Ubermitteln des Feedbacks an das Suchverfahren
	\newline
	\item Unterstützung der Bildformate JPEG, PNG und BMP
	\item Unterstützung von Videos in Form von Einzelbildern in untert\"utzten Bildformaten

	\subsection{Wunsch}
	\item Wählen einer Suchvorlagen, die nicht in den gewählten Datensätzen enthalten ist
	\newline
	\item Anhalten der Suche 	% noetig ?
	\item Fortsetzen der angehaltenen Suche 	% noetig ?
	\item Rückkehr in eine frühere Suche % unklar
	\newline
	\item Löschen von Ergebnissen % unklar
	\item Abspielen eines Benachrichtigungstons bei Abschluss einer Suche
	\item Ein-/Abschalten des Benachrichtigungstons
	\item Abspielen von Videos (mit Ausgabe der Audiospur)
	\item Durchführen einer weiteren Suche in den Ergebnissen einer bereits beendeten Suche
	\newline
	\item Unterstützung der Videoformate AVI, FLK, MKV, MP4
	\item Wechsel in einen Vollbildmodus für Präsentation
%	\newline
%	\item Duplizieren des aktuellen Suchkontextes
%	\item Wechsel des aktuellen Suchkontextes %unklar
%	\item nicht-fl\"uchtiges Speichern des Suchkontexts
%	\item Laden eines gespeicherten Suchkontextes
%	\item Verkn\"upfen (bilden von Vereinigungs- oder Schnittmenge) von Suchergebnissen verschiedener Suchkontexte
\end{enumerate}
