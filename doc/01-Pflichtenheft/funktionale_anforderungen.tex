% !TeX spellcheck = de_DE_frami
\begin{enumerate} [label=\bfseries /F \arabic*0/]
	\subsection{Pflicht}
	\item Starten des Programms
	\item Beenden des Programms
	\item Auswahl der Sprache
%	\item Er\"offnen eines neuen Suchkontexts
%	\item Er\"offnen eines neuen \glslink{Suchkontext}{Suchkontexts}
	\item Ausw\"ahlen der Datens\"atze
	\item Ausw\"ahlen eines Bildes oder Videos aus den gewählten Datensätzen als Suchvorlage
	\item Einschränkung der Suchvorlage auf einen benutzerdefinierten rechteckigen Bereich
	\item Auswahl des Suchverfahrens
	\item Starten der Suche
	\item Sanduhr w\"arhend des Suchvorgangs anzeigen 
	\item Abbrechen der Suche
	\item Anzeigen der \"Ubersicht der Suchergebnisse
	\item Auswahl und Anzeige eines einzelnen Suchergebnisses
	\item Abspielen eines Videos aus Einzelbildern
	\item Einstellen eines Feedbacks zu einem Schergebnis
	\item \"Ubermitteln des Feedbacks an das Suchverfahren
	\item Help-Menü mit Anweisungen für den Benutzer
	\item About-Menü mit Informationen über das Programm
	\item Unterstützung der Bildformate JPEG, PNG und BMP
	\item Unterstützung von Videos in Form von Einzelbildern in untert\"utzten Bildformaten
	\subsection{Wunsch}
	\item Anhalten der Suche 	% noetig ?
	\item Fortsetzen der angehaltenen Suche 	% noetig ?
	\item Zurück in eine frühere Suche gehen % unklar
	\item Ergebnisse löschen % unklar
	\item Abspielen eines Benachrichtigungstons bei Abschluss einer Suche
	\item Ein-/Abschalten des Benachrichtigungstons
	\item Abspielen von Videos
	\item Duplizieren des aktuellen Suchkontextes
	\item Wechsel des aktuellen Suchkontextes %unklar
	\item nicht-fl\"uchtiges Speichern des Suchkontexts
	\item Laden eines gespeicherten Suchkontextes
	\item Verkn\"upfen (bilden von Vereinigungs- oder Schnittmenge) von Suchergebnissen verschiedener Suchkontexte
	\item Unterstützung der Videoformate AVI, FLK, MKV, MP4
	\item Durchführen einer Suche in den Ergebnissen einer beendeten Suche
	\item Wechsel in einen Vollbildmodus für Präsentation
	\item Wählen einer Suchvorlagen, die nicht in den gewählten Datensätzen enthalten ist
\end{enumerate}
