% !TeX spellcheck = de_DE_frami
\subsection{Pflichtanforderungen}
\begin{description}
	\item[\req{F 10}] Starten des Programms
	\item[\req{F 20}] Beenden des Programms
	\item[\req{F 30}] Automatisches Erkennen verfügbarer Suchverfahren
	\item[\req{F 40}] Anzeigen eines Hilfe-Dialogs mit Hinweisen zur Benutzung des Programms
	\item[\req{F 50}] Anzeigen eines About-Dialogs mit Informationen zum Programm
	\newline
%	\item[\req{F 0}] Er\"offnen eines neuen Suchkontexts
%	\item[\req{F 0}] Er\"offnen eines neuen \glslink{Suchkontext}{Suchkontexts}
	\item[\req{F 60}] Wiederanzeigen der Ergebnisse einer vorherigen Suche (siehe auch \req{F 310})
	\item[\req{F 70}] Anzeigen einer Übersicht der gespeicherten Suchergebnisse
	\item[\req{F 80}] Anzeigen einer Bibliothek von Datensätzen
	\item[\req{F 90}] Hinzufügen von Datensätzen zur Bibliothek
	\item[\req{F 100}] Ausw\"ahlen eines Datensatzes aus der Bibliothek
	\item[\req{F 110}] Anzeigen einer Übersicht der Bilder bzw. Videos des gewählten Datensatzes
	\item[\req{F 115}] Unterstützung der Bildformate JPEG, PNG und BMP
	\item[\req{F 120}] Anzeigen einer größeren Darstellung eines ausgewählten Bildes bzw. Videos des Datensatzes
	\item[\req{F 130}] Abspielen eines Videos aus Einzelbildern
	\item[\req{F 135}] Buttons zur Auswahl des vorherigen oder nächsten Bildes
	\item[\req{F 140}] Ausw\"ahlen eines Bildes bzw. Videos aus dem gewählten Datensatz als Suchvorlage
	\item[\req{F 150}] Anzeigen annotierter Bildbereiche
	\item[\req{F 160}] Auswahl eines annotierten Bildbereiches als Suchvorlage
	\item[\req{F 170}] Einschränkung der Suchvorlage auf einen benutzerdefinierten rechteckigen Bereich (falls keine Annotation gewählt wurde)
	\item[\req{F 180}] Auswahl eines für ggf. gewählte Annotation geeigneten Suchverfahrens
	\item[\req{F 190}] Anzeigen einer Beschreibung für die Suchverfahren
	\item[\req{F 200}] Festlegen der für das gewählte Suchverfahren spezifischen Parameter
	\item[\req{F 210}] Anzeigen sämtlicher vorgenommener Einstellungen zur aktuellen Suche, um Überprüfung durch den Benutzer zu ermöglichen
	\item[\req{F 220}] Ändern sämtlicher vorgenommener Einstellungen nach der Überprüfung
	\newline
	\item[\req{F 230}] Starten der Suche
	\item[\req{F 240}] Anzeigen einer Fortschrittsanimation während des Suchvorgangs
	\item[\req{F 250}] Abbrechen der Suche
	\newline
	\item[\req{F 260}] Anzeigen einer \"Ubersicht der Suchergebnisse  
	\item[\req{F 290}] Einstellen eines Feedbacks zu einem Suchergebnis und Starten einer weiteren Suche mit diesem Feedback und dem selben Suchverfahren
	\item[\req{F 310}] nicht-flüchtiges Speichern der Suchergebnisse (siehe auch \req{F 60})
	\item[\req{F 320}] Loggen von Informationen, Warnungen, Fehlern und Debugmeldungen
	 
\end{description}

\subsection{Wunschanforderungen}
\begin{description}
	\item[\req{FW 10}] nicht-flüchtige Auswahl der Übersetzungen der GUI in Deutsch und Englisch
	\newline
	\item[\req{FW 20}] Wählen einer Suchvorlage, die nicht in den gewählten Datensätzen enthalten ist (erweitert \req{F 140})
	\item[\req{FW 25}] Wählen von mehreren Datensätzen in denen gesucht wird (erweitert \req{F 100})
	\newline
	\item[\req{FW 30}] Abspielen eines Benachrichtigungstons bei Abschluss einer Suche
	\item[\req{FW 40}] Ein-/Abschalten des Benachrichtigungstons
	\item[\req{FW 50}] Abspielen von Videos (mit Ausgabe der Audiospur)
	\item[\req{FW 60}] Durchführen einer weiteren Suche in den Ergebnissen einer bereits beendeten Suche mit anderem Suchverfahren \req{F 290}
	\newline
	\item[\req{FW 70}] Wechsel in einen Vollbildmodus für Präsentation
	
	\item[\req{FW 80}] Zoomen und Scrollen in der größeren Darstellung in der Datensatzauwahl
	\item[\req{FW 90}] Generierung eines zufälligen oder festgelegten Vorschaubildes für den Datensatz
	\item[\req{FW 100}] Anzeigen einer größeren Darstellung eines ausgewählten Suchergebnisses
%	\newline
%	\item[\req{F 0}] Duplizieren des aktuellen Suchkontextes
%	\item[\req{F 0}] Wechsel des aktuellen Suchkontextes %unklar
%	\item[\req{F 0}] nicht-fl\"uchtiges Speichern des Suchkontexts
%	\item[\req{F 0}] Laden eines gespeicherten Suchkontextes
%	\item[\req{F 0}] Verkn\"upfen (bilden von Vereinigungs- oder Schnittmenge) von Suchergebnissen verschiedener Suchkontexte
\end{description}
\pagebreak