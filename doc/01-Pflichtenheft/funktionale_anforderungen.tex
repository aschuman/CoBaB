% !TeX spellcheck = de_DE_frami
\subsection{Pflichtanforderungen}
\begin{enumerate} [label=\bfseries /F \arabic*0/, leftmargin=*]
	\item Starten des Programms \label{f:programmstart}
	\item Beenden des Programms \label{f:beenden}
	%\item Programmführung auf deutsch % keine Funktion
	\item Automatisches Erkennen verfügbarer \gls{Suchverfahren} \label{f:erkennen_suchverfahren}
	\item Übergeben eines Standardordners für die Bibliotheksdatensätze über die Kommandozeile \label{f:standardordner_übergeben}
	\item Anzeigen eines Hilfe-\glslink{Dialog}{Dialogs} mit Hinweisen zur Benutzung des Programms \label{f:hilfe}
	\item Anzeigen eines About-\glslink{Dialog}{Dialogs} mit Informationen zum Programm \label{f:about}
	\item Rückkehr zur Bibliothek \label{f:rückkehr_zu_bibliothek}
	%\item Zurück-Funktion, um zur vorherigen Seite zurückzukehren % siehe f:einstellungen_nach_ueberpruefung_aendern
	\newline
 
	\item Anzeigen einer Bibliothek von Datensätzen, wie in Abbildung \ref{fig:bibliothek} zu sehen ist \label{f:bibliothek_anzeigen}
	\item Auswählen eines Datensatzes, der nicht in der Bibliothek enthalten ist, wie in Abbildung \ref{fig:filedialog} zu sehen ist \label{f:datensatz_hinzufuegen}
	\item Ausw\"ahlen eines Datensatzes aus der Bibliothek \label{f:auswahl_eines_datensatzes}
	\item Anzeigen einer Übersicht der Bilder bzw. Videos des gewählten Datensatzes, wie in Abbildung \ref{fig:fotoanzeiger} zu sehen ist \label{f:uebersicht_anzeigen}
	\item Unterstützung der Bildformate JPEG, PNG und BMP \label{f:bildformate}
	\item Anzeigen einer größeren Darstellung eines ausgewählten Bildes bzw. Videos des Datensatzes \label{f:anzeigen_groessere_darstellung}
	\item Abspielen eines Videos aus Einzelbildern \label{f:video_abspielen}
	\item Auswählen des vorherigen oder nächsten Bildes bzw. Videos für die große Ansicht \label{f:vorheriges_naechstes}
	\item Ausw\"ahlen eines Bildes bzw. Videos aus dem gewählten Datensatz als Suchvorlage \label{f:bildauswahl}
	\item Anzeigen \glslink{Annotation}{annotierter Bildbereiche} \label{f:annotation_anzeigen}
	\item Auswahl eines \glslink{Annotation}{annotierten Bildbereiches} als Suchvorlage \label{f:annotation_auswaehlen}
	\item Einschränkung der Suchvorlage auf einen benutzerdefinierten rechteckigen Bereich (falls keine \gls{Annotation} gewählt wurde) \label{f:bereich_auswaehlen}
	\item Auswahl eines für die gewählte Suchvorlage geeigneten \glslink{Suchverfahren}{Suchverfahrens} \label{f:auswahl_suchverfahren}
	\item Anzeigen einer Beschreibung für die \gls{Suchverfahren} \label{f:beschreibung_suchverfahren}
	\item Festlegen der für das gewählte \gls{Suchverfahren} spezifischen Parameter, wie in Abbildung \ref{fig:parameterauswahl} zu sehen ist \label{f:parameterwahl}
	\item Anzeigen sämtlicher vorgenommener Einstellungen zur aktuellen Suche, um eine Überprüfung durch den Benutzer zu ermöglichen, wie in Abbildung \ref{fig:überprüfung} zu sehen ist \label{f:ueberpruefung}
	\item Ändern beliebiger zu einer Suche vorgenommener Einstellungen \label{f:einstellungen_nach_ueberpruefung_aendern}
	\newline
	\item Starten der Suche \label{f:suche_starten}
	\item Anzeigen einer Fortschrittsanimation während des Suchvorgangs \label{f:fortschrittsanimation}
	\newline
	\item Anzeigen einer sortierten \"Ubersicht der Suchergebnisse, wie in Abbildung \ref{fig:suchergebnisse} zu sehen ist \label{f:suchergebnisse_anzeigen}
	\item Optionales Einstellen eines \glslink{Feedback}{Feedbacks} zu einem Suchergebnis und Starten einer weiteren Suche mit diesem \gls{Feedback} und dem selben \gls{Suchverfahren} \label{f:feedback}
	
	\item Loggen von Informationen, Warnungen, Fehlern und Debugmeldungen \label{f:loggen}
	 
\end{enumerate}

\subsection{Wunschanforderungen}
\begin{enumerate} [label=\bfseries /FW \arabic*0/, leftmargin=*]
	\item Nicht-flüchtige Auswahl der Übersetzungen der \gls{GUI} in Deutsch und Englisch \label{fw:sprache}
	\newline
	\item Wählen von mehreren Datensätzen, in denen gesucht wird (erweitert \ref{f:auswahl_eines_datensatzes}) \label{fw:mehrere_datensaetze_waehlen}
	\item Generierung eines zufälligen oder festgelegten Vorschaubildes für den Datensatz \label{fw:vorschaubild}
	\newline
	\item Abspielen eines Benachrichtigungstons bei Abschluss einer Suche \label{fw:signalton}
	\item Ein-/Abschalten des Benachrichtigungstons \label{fw:signalton_einaus}
	\item Abspielen von Videos aus Videodateien (mit Ausgabe der Audiospur) \label{fw:echtes_video_abspielen}
	\newline
	\item Festlegung eines Wertes zwischen 0 und 10 als \gls{Feedback} \label{fw:zehner_feedback}
	\item Durchführen einer weiteren Suche in den Ergebnissen einer bereits beendeten Suche mit anderem \gls{Suchverfahren} (siehe auch \ref{f:feedback}) \label{fw:weitere_suche}
	%\item Duchführen von zwei verschiedenen Suchverfahren nacheinander und Anzeigen der Schnitt- oder Vereinigungsmenge als Ergebnis % siehe vorherige Funktion
	\newline
	\item Wechsel in einen Vollbildmodus für Präsentation \label{fw:präsentation}
	\item Zoomen und Scrollen in der größeren Darstellung im Fotoanzeiger oder Videoplayer \label{fw:zoom_scroll}
	\item Anzeigen einer größeren Darstellung eines ausgewählten Suchergebnisses \label{fw:groesseres_suchergebnis}
	\newline
	\item Nicht-flüchtiges Speichern der zuletzt verwendeten Datensätze \label{fw:speichern_historie}
	\item Nicht-flüchtiges Speichern der \gls{Lesezeichen} (siehe auch \ref{fw:lesezeichen}) \label{fw:speichern_lesezeichen}
	\item Automatisches nicht-flüchtiges Speichern der \gls{Suchchronik} (siehe auch \ref{fw:chronik}) \label{fw:speichern_chronik}
	\item Anzeigen der Historie der Datensätze (siehe auch \ref{fw:speichern_historie}) \label{fw:historie_uebersicht}
	\item Anzeigen einer Übersicht der \gls{Lesezeichen} \label{fw:lesezeichen_uebersicht}
	\item Anzeigen einer Übersicht der \gls{Suchchronik} \label{fw:chronik_uebersicht}
	\item Wiederanzeigen der Ergebnisse eines \glslink{Lesezeichen}{Lesezeichens} (siehe auch \ref{fw:speichern_lesezeichen}) \label{fw:lesezeichen}
	\item Wiederanzeigen der Ergebnisse der \gls{Suchchronik} (siehe auch \ref{fw:speichern_chronik}) \label{fw:chronik}
	\newline
	\item Abbrechen der Suche mit Weiterführung der Oberfläche \label{fw:suche_abbrechen}
\end{enumerate}
\pagebreak
