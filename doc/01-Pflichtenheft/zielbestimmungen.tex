CoBaB ermöglicht dem Benutzer anhand eines ausgewählten Bildes oder Videos eine inhaltsbasierte Suche in Bild- oder Videodaten durchzuführen. Als Ergebnis liefert CoBaB eine Auswahl ähnlicher Bilder oder Videos, die durch Eingabe von \gls{Feedback} verfeinert werden kann.
\subsection{Musskriterien}
\begin{enumerate} [label=\bfseries /MK \arabic*0/, leftmargin=*]
%Anwendung
\item Der Benutzer wird auf Deutsch durch das Programm geführt.
\item Es gibt ein Startmenü mit einer Datensatzauswahl, im Folgenden Bibliothek genannt.
\item Das Programm verfügt über einen Fotoanzeiger für die ausgewählte Bildvorlage.
\item Das Programm verfügt über einen Videoplayer für Videos, die in Form von Einzelbildern vorliegen.
\item Der Benutzer kann nach einem ganzen Bild oder Video suchen lassen.
\item Der Benutzer kann einen Bereich des Bildes oder Videos mit einem Rahmen kennzeichnen und nach diesem Bereich suchen lassen.
\item Der Benutzer kann nach \glslink{Annotation}{Annotationen} im Bild oder Video suchen lassen.
\item Der Benutzer kann Parameter für das \gls{Suchverfahren} festlegen.
\item Nachdem der Suchvorgang beendet ist, werden die Ergebnisse auf der Benutzeroberfläche in Form von Miniaturbildern angezeigt.
\item Die Suchergebnisse sind in einer Rangliste von meist zutreffend bis weniger zutreffend angeordnet.
\item Der Benutzer kann jedes Bild des Suchergebnisses als zutreffend oder nicht zutreffend bewerten, im Folgenden \gls{Feedback} genannt.
\item Nachdem der Benutzer die Suchergebnisse bewertet hat, wird diese Bewertung beim Start der nächsten Suche an den \glslink{Suchverfahren}{Suchalgorithmus} zurückgemeldet.
\item Der Benutzer kann einen neuen Suchvorgang starten, ohne die Ergebnisse zu bewerten.
\item Der Benutzer kann durch die zurück-Funktion die letzten Schritte rückgängig machen.
\item Das Programm kann jederzeit durch den Benutzer beendet werden, insbesondere während die Suche läuft.
% Schnittstellen Daten und Suchverfahren
\item Es gibt eine Schnittstelle für die Anwendung der gegebenen \gls{Suchverfahren}.
\item Es existiert eine einheitliche Schnittstelle für die Bild- und Videodateien.
\item Das Programm unterstützt mehrere gängige Bildformate.
\item Es kann ein Standardordner für die Datensatzbibliothek per Kommandozeile übergeben werden.
%technisch
\item Das Programm läuft unter Microsoft Windows 8 und Linux (Ubuntu 14.04 und Fedora 22).
\end{enumerate}
\subsection{Wunschkriterien}
\begin{enumerate} [label=\bfseries /WK \arabic*0/, leftmargin=*]
\item Der Benutzer kann auch auf Englisch durch das Programm geführt werden.
\item Das Programm unterstützt auch Videodateien im Gegensatz zu Videos aus Einzelbildern.
\item Der Benutzer kann sich die Bilder/ Videos des gewählten Datensatzes und die Suchergebnisse als Vollbild anzeigen lassen.
\item Der Benutzer kann zwei unterschiedliche \gls{Suchverfahren} nacheinander verwenden. Als Suchergebnis wird die Schnittmenge oder die Vereinigungsmenge angezeigt.
\item Der Benutzer kann den Suchvorgang stoppen.
\item Nachdem der Suchvorgang beendet ist, wird der Benutzer mit einem Ton benachrichtigt. Diesen kann er an- oder ausschalten.
\item Der Benutzer kann für die Suchergebnisse einen Wert zwischen 0 und 10 als \gls{Feedback} festlegen.
\item Der Benutzer kann das Suchergebnis als \gls{Lesezeichen} speichern.
\item Die letzten Suchergebnisse werden in der \gls{Suchchronik} gespeichert.
\item Das Programm bietet einen Vollbildmodus für Präsentationen an.
\item Die Historie der zuletzt verwendeten Datensätze wird erinnert.
\end{enumerate}
\subsection{Abgrenzungskriterien}
\begin{enumerate} [label=\bfseries /AK \arabic*0/, leftmargin=*]
\item Das Programm muss nicht unter Microsoft Windows XP und Apple OS X laufen. 
\item Es ist nicht für Tablets und Handys vorgesehen.
\item Zur Laufzeit können keine \gls{Suchverfahren} hinzugefügt werden.
\item Es wird kein QML verwendet.
\end{enumerate}
\pagebreak
