Der Media Browser ermöglicht dem Benutzer anhand eines ausgewählten Bildes oder Videos eine inhaltsbasierte Suche in Bild- oder Videodaten durchzuführen. Als Ergebnis liefert der Browser eine Auswahl ähnlicher Bilder oder Videos.
\subsection{Musskriterien}
\begin{enumerate} [label=\bfseries /MK \arabic*0/]
%Anwendung
\item Der Benutzer wird auf Deutsch durch das Programm geführt.
\item Es gibt eine Startmenü und eine Datensatzauswahl.
\item Die ausgewählte Bildvorlage und die Suchergebnis-Bilder werden durch einen Fotoanzeiger dargestellt.
\item Der Benutzer kann sich die Bilder der Datensatzauswahl und die Suchergebnisse als Vollbild anzeigen lassen.
\item Das Programm verfügt über einen Videoplayer.
\item Der Benutzer kann festlegen, ob nach Bildern oder Videos, die als Einzelbilder vorliegen, gesucht wird.
\item Der Benutzer hat die Möglichkeit ein Bild aus der Datenbibliothek auszuwählen.
\item Der Benutzer kann nach einem ganzen Bild suchen lassen.
\item Der Benutzer kann einen Bereich des Bildes mit einem Rahmen kennzeichnen(= Bildannotation) und diesen Bereich suchen lassen.
\item Der Benutzer startet durch Anklicken der Suchtaste einen neuen Suchvorgang.
\item Nachdem der Suchvorgang beendet ist, werden die Ergebnisse auf der Benutzeroberfläche in Form von Miniaturbildern angezeigt.
\item Die Suchergebnisse sind in einer Rangliste von meist zutreffend bis weniger zutreffend angeordnet.
\item Der Benutzer kann jedes Bild des Suchergebnisses als zutreffend oder nicht zutreffend bewerten.
\item Nachdem der Benutzer die Suchergebnisse bewertet hat, wird diese Bewertung beim Start der nächsten Suche an den Suchalgorithmus zurückgemeldet.
\item Der Benutzer kann einen neuen Suchvorgang starten, ohne die Ergebnisse zu bewerten.
\item Der Benutzer kann durch die undo-Funktion die letzten vier Schritte rückgängig machen.
\item Das Programm kann jederzeit durch den Benutzer beendet werden.
% Schnittstelle Daten
\item Es existiert eine einheitliche Schnittstelle für die Bild- und Videodateien (in Form von einzelnen Bilddateien).
\item Das Programm unterstützt mehrere Bildformate.
%Schnittstelle Suchverfahren
\item Es gibt eine Schnittstelle für die Anwendung der gegebenen Suchverfahren.
%technisch
\item Das Programm läuft unter Microsoft Windows und Linux.
\end{enumerate}
\subsection{Wunschkriterien}
\begin{enumerate} [label=\bfseries /WK \arabic*0/]
\item Der Benutzer kann in weiteren Sprachen (englisch, bulgarisch) durch das Programm geführt werden.
\item Das Programm soll auch in Videos, die in AVI, FLV, MKV, MP4 kodiert sind, nach Bildern oder Bildausschnitten suchen können.
\item Der Benutzer kann zwei Suchverfahren parallel verwenden. Als Suchergebnis wird die Schnittmenge oder die Vereinigungsmenge angezeigt.
\item Der Benutzer kann den Suchvorgang stoppen.
\item Nachdem der Suchvorgang beendet ist, wird der Benutzer mit einem Ton benachrichtigt.
\item Der Benutzer kann den Benachrichtigungston an- oder ausschalten.
\item Der Benutzer kann für die Suchergebnisse einen Wert zwischen 0 und 10 als Feedback festlegen.
\item Der Benutzer kann das Suchergebnis speichern.
\item ? (zu kurz?, nur beim Messestand) Das Programm soll nicht einfrieren. Wenn der Benutzer eine Minute inaktiv ist, kehrt das Programm in den Startzustand zurück.
\item Der Benutzer kann die Benutzeroberfläche minimieren.  
\end{enumerate}
\subsection{Abgrenzungskriterien}
\begin{enumerate} [label=\bfseries /AK \arabic*0/]
\item Das Programm muss nicht unter Microsoft Windows XP und Apple OS X laufen. 
\item Es ist nicht für Tablets und Handys vorgesehen.
\end{enumerate}
