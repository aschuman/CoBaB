Der Media Browser ermöglicht dem Benutzer anhand eines ausgewählten Bildes oder Videos eine inhaltsbasierte Suche in Bild- oder Videodaten durchzuführen. Als Ergebnis liefert der Browser eine Auswahl ähnlicher Bilder oder Videos.
\subsection{Musskriterien}
\begin{enumerate} [label=\bfseries /MK \arabic*0/]
\item ? Es existiert eine einheitliche Schnittstelle für die Bild- und Videodateien (in Form von einzelnen Bilddateien) sowie eine Schnittstelle für die Anwendung der gegebenen Suchverfahren.
\item Das Programm läuft unter Microsoft Windows 7/8/10/Vista, Fedora 20, Ubuntu 14 und Debian 7.
\item Der Benutzer wird auf Deutsch oder Englisch durch das Programm geführt.
\item Nachdem das Programm geladen wurde, wird eine menügesteuerte Benutzerführung (Startmenü) angezeigt.
\item Der Benutzer wählt ein Bild aus der Datenbibliothek aus.
\item Die ausgewählte Bildvorlage und die Suchergebnis-Bilder werden durch einen Fotoanzeiger dargestellt.
\item Der Benutzer kann ein ganzes Bild suchen lassen.
\item Der Benutzer kann einen Bereich des Bildes mit einem Rahmen kennzeichnen(= Bildannotation) und diesen Bereich suchen lassen.
\item Verschiedene Suchverfahren ermöglichen es, nach Personen oder Objekten zu suchen.
\item Die Suchergebnisse sind in einer Rangliste von meist zutreffend bis weniger zutreffend angeordnet.
\item Der Benutzer startet durch Anklicken der Suchtaste einen neuen Suchvorgang.
\item Der Benutzer legt fest, ob nach Bildern oder Videos, die als Einzelbilder vorliegen, gesucht wird.
\item Der Benutzer kann durch die undo-Funktion die letzten vier Schritte rückgängig machen.
\item Das Programm unterstützt Bilder mit den Datenformaten JPEG, PNG und BMP.
\item ? Der Benutzer gibt die Merkmale in Form von Worten (jedes Merkmal ein Wort) in das Eingabefenster.
\item ? Der Benutzer kann durch die GUI browsen und die Daten für die Suche bestimmen. Wenn der dies nicht macht, wird der Default Ordner verwendet.
\item Der Benutzer kann jedes Bild des Suchergebnisses als zutreffend oder nicht zutreffend bewerten.
\item Nachdem der Suchvorgang beendet ist, werden die Ergebnisse auf der Benutzeroberfläche in Form von Miniaturbildern angezeigt.
\item Der Benutzer kann sich die Suchergebnisse als Vollbild anzeigen lassen.
\item Der Benuzter kann einen neuen Suchvorgang starten, ohne die Ergebnisse zu bewerten.
\item Nachdem der Benutzer die Suchergebnisse bewertet hat, wird diese Bewertung an den Suchalgorithmus zurückgemeldet.
\item Das Programm kann jederzeit durch den Benutzer beendet werden.
\end{enumerate}
\subsection{Wunschkriterien}
\begin{enumerate} [label=\bfseries /WK \arabic*0/]
\item Der Benutzer kann in weiteren Sprachen durch das Programm geführt werden.
\item Der Benutzer kann für die Suchergebnisse ein Ranking festlegen. 
\item Das Programm soll über einen Videoplayer verfügen.
\item ? (sind das relevante Formate?) Das Programm soll auch in Videos, die in AVI, FLV, MKV, MP4 kodiert sind, nach Bildern oder Bildausschnitten suchen können.
\item ? Der Benutzer kann den Suchvorgang anhalten.
\item ? Der Benutzer kann einen angehaltenen Vorgang kann wieder fortsetzen.
\item Nachdem der Suchvorgang beendet ist, soll der Benutzer mit einem Ton benachrichtigt werden.
\item Der Benutzer kann den Benachrichtigungston an- oder ausschalten.
\item ? (zu kurz?) Das Programm soll nicht einfrieren. Wenn der Benutzer eine Minute inaktiv ist, kehrt das Programm in den Startzustand zurück.
\item Der Benutzer kann die Benutzeroberfläche minimieren. 
\item Der Benutzer kann zwei Suchverfahren parallel verwenden. Als Suchergebnis wird die Schnittmenge oder die Vereinigungsmenge angezeigt.
\item Der Benutzer kann das Suchergebnis speichern. 
\end{enumerate}
\subsection{Abgrenzungskriterien}
\begin{enumerate} [label=\bfseries /AK \arabic*0/]
\item Das Programm läuft nicht unter Microsoft Windows XP und  Apple OS X. Es ist auch für Tablets und Handys nicht vorgesehen.
\end{enumerate}
