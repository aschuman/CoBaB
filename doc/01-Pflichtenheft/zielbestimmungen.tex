Das Produkt soll dem Benutzer ermöglichen für ein eingegebenes Bild oder Video gestützt durch ein benutzerfestgelegtes Verfahren automatisch eine Auswahl inhaltlich ähnlicher Bilder oder Videos zu erhalten.
\subsection{Musskriterien}
\begin{itemize}
\item Das Programm läuft unter Microsoft Windows 7/8/10/Vista, Fedora 20, Ubuntu 14, Debian 7
\item Die Programmsprache ist English / Deutsch 
\item Nachdem das Programm geladen wurde, wird eine GUI angezeigt
\item Aus Datenbibliothek kann ein Bild gewählt werden
\item GUI hat Foto-Viewer
\item Benutzer kann ganzes Bild oder ein Teil davon als Suchmuster festlegen
\item Benutzer kann Bildannotation manuell erstellen
\item Verschiedene Suchverfahren
\item Suchergebnisse von meist bis wenigsten Zutreffendes angeordnet
\item Der Benutzer kann durch anklicken an die ''Search''-Taste einen neuen Suchvorgang zu starten
\item Der Benutzer kann festlegen, nach Bilder oder Videodaten in form von Einzelbilder zu suchen
\item Undo-Funktion fuer letzten 3 Schritten
\item Das Programm unterstützt Bilder mit Datenformat JPG, PNG, BMP
\item Das Programm fragt den Nutzer, nach welche Merkmale zu suchen durch ein Dialogfenster
\item Der Benutzer gibt die Merkmale in form von Worte (jedes Merkmal ein Wort) in das gezeige Eingabefenster.
\item Der Benutzer kann durch die GUI browsen und die Daten für die Suche bestimmen. Wenn der dies nicht macht, wird der Default  Ordner verwendet.
\item Benutzer-Feedback-Funktion : Bewerten die Bilder mit Punkten / Ja-Nein
\item Benutzer-Feedback dient zu Verbesserung der Suchergebnis
\item Nachdem der Suchvorgang beendet ist, werden die Ergebnisse auf der GUI in Form von Maniaturbilder angezeigt. 
\item Der Benutzer kann Feedback geben. Für jedes angezeigt Bild kann er bestimmen, ob die Suchergebnisse seiner Suchbegriffe entsprechen. 
\item Der Benuzter kann einen neuen Suchvorgang starten, ohne die Ergebnisse zu bewerten.
\item Nachdem der Benutzer auf Ok clickt, wird die Bewertung der Suchergebnisse gespeichert
\item Der Benutzer kann das Programm jederzeit durch Anclicken auf das X-Symbol auf der Taskleiste oder in File/Exit beenden. 
\item Ein laufender Suchvorgang wird vor dem Beenden des Programm gehalten und terminiert.
\end{itemize}
\subsection{Wunschkriterien}
\begin{itemize}
\item Der Benutzer kann die Sprache in Options/ Language ändern
\item Videoplayer
\item Das Programm sollte auch Videodaten, die in AVI, FLV, MKV, MP4 kodiert sind, nach Merkmale suchen können
\item Der Benutzer kann den Suchvorgang anhalten
\item Der Benutzer kann einen angehaltener Vorgang kann wiederfortsetzen
\item Nachdem der Suchvorgang beendet ist, sollte der Benutzer mit einem Ton benachrichtigt werden
\item Der Benutzer kann den Benachrichtigungston in Options/Notification an- und ausschalten 
\item Das Program sollte nicht frieren. Wenn ein Suchvorgang nach 1 Minute inaktiv, sollte der Vorgang unterbrochen werden und eine Fehlermeldung gezeigt werden.
\item Der Benutzer kann die GUI minimieren
\item Der Benutzer kann die GUI maximieren (Fullscreen Mode) 
\item Der Benutzer kann 2 Suchverfahren verwenden. Als Ergebnis wird die Vereinigugs/Durschschnitt angezeigt
\item Der Benutzer kann Bookmark (Das Ergebnis der Suche) speichern 
\end{itemize}
\subsection{Abgrenzungskriterien}
\begin{itemize}
\item Das Programm ist unter Windows Xp oder Tablet-OS, Handy-OS nicht lauffähig
\end{itemize}
