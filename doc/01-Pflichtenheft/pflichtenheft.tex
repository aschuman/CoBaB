\documentclass[parskip=full]{scrartcl}
\usepackage[utf8]{inputenc}
\usepackage[T1]{fontenc}
\usepackage[ngerman]{babel}
\usepackage{enumerate}
\usepackage{enumitem}
\usepackage{hyperref}
%\usepackage[toc]{glossaries}

\setlist[enumerate]{itemsep=-2.5mm}

\title{ContentBasedBrowser -CoBaB- \\ Pflichtenheft}
\hypersetup{
    colorlinks,
    citecolor=black,
    filecolor=black,
    linkcolor=black,
    urlcolor=black
}
%\makeglossaries
%\newglossaryentry{Suchkontext}
{
name=Suchkontext,
description={enth\"alt und ist definiert durch genau eine (annotierte) Suchvorlage, genau ein Suchverfahren, mindestens einen Datensatz und ggf. die entsprechenden Suchergebnisse}
}

\newglossaryentry{Annotation}
{
name=Annotation,
description={ist ein in den Datensätzen vordefinierter Bereich eines Bildes oder Videos}
}


\begin{document}
\begin{titlepage}
\title{ContentBasedBrowser -CoBaB- \\ Pflichtenheft}
\author{Anja Blechinger, Marie Bommersheim, Georgi Georgiev,\\ Tung Nguyen, Vincent Winkler, Violina Zhekova}
\date{November 2015}
\maketitle
\begin{verbatim}
























\end{verbatim}
\begin{tabular}{l l}
Projekt: & Media Browser zur inhaltsbasierten Suche in Bild- und Videodaten\\
Auftraggeber: & Arne Schumann,\\
 & Fraunhofer Institut für Optronik, Systemtechnik und Bildauswertung, Karlsruhe\\
\end{tabular}
\thispagestyle{empty}
\end{titlepage}
\setcounter{page}{1}

\tableofcontents
\pagebreak

\section{Einleitung}
Wer hat nicht schon einmal das Internet nach Bildern von sich oder Freunden durchsucht? Dabei wird häufig textbasierte Suche verwendet. Man tippt also einen Namen ein und erhält alle Bilder, auf denen die Person sein könnte, weil der Name der gesuchten Person im Text nebenan erwähnt wird. 
\newline
In Zukunft wird allerdings die inhaltsbasierte Suche immer wichtiger. Die Menge an verfügbaren Daten nimmt stetig zu und es treten Probleme mit der Datenhandhabung auf. Ohne inhaltsbasierte Suche ist eine Navigation durch die Datenbestände sehr aufwändig.
Nach einem Urlaub in London beispielsweise hat man sehr viele Urlaubsfotos. Sieht man die Fotos durch und entdeckt eine Aufnahme vom London Eye und möchte sich alle weiteren Fotos mit diesem Motiv anzeigen lassen, dann ist die inhaltsbasierte Suche hilfreich.
\newline
Unser Projekt soll dies nun basierend auf Datensätzen der Anwender und mit den Suchverfahren des Fraunhofer IOSB ermöglichen.
\newline
Dieses Pflichtenheft soll dabei einen ersten Einblick in die Anwendung liefern, indem es u.a. die Muss- und Wunschkriterien, die Produktanforderungen und die Benutzerschnittstelle erläutert.
\pagebreak

\section{Zielbestimmungen}
CoBaB ermöglicht dem Benutzer anhand eines ausgewählten Bildes oder Videos eine inhaltsbasierte Suche in Bild- oder Videodaten durchzuführen. Als Ergebnis liefert CoBaB eine Auswahl ähnlicher Bilder oder Videos, die durch Eingabe von \gls{Feedback} verfeinert werden kann.
\subsection{Musskriterien}
\begin{enumerate} [label=\bfseries /MK \arabic*0/, leftmargin=*]
%Anwendung
\item Der Benutzer wird auf Deutsch durch das Programm geführt.
\item Es gibt ein Startmenü mit einer Datensatzauswahl, im Folgenden Bibliothek genannt.
\item Das Programm verfügt über einen Fotoanzeiger für die ausgewählte Bildvorlage und die Suchergebnisse.
\item Das Programm verfügt über einen Videoplayer für Videos, die in Form von Einzelbildern vorliegen.
\item Der Benutzer kann nach einem ganzen Bild oder Video suchen lassen.
\item Der Benutzer kann einen Bereich des Bildes oder Videos mit einem Rahmen kennzeichnen und nach diesem Bereich suchen lassen.
\item Der Benutzer kann nach Annotationen im Bild oder Video suchen lassen.
\item Der Benutzer kann Parameter für das Suchverfahren festlegen.
\item Nachdem der Suchvorgang beendet ist, werden die Ergebnisse auf der Benutzeroberfläche in Form von Miniaturbildern angezeigt.
\item Die Suchergebnisse sind in einer Rangliste von meist zutreffend bis weniger zutreffend angeordnet.
\item Der Benutzer kann jedes Bild des Suchergebnisses als zutreffend oder nicht zutreffend bewerten, im Folgenden \gls{Feedback} genannt.
\item Nachdem der Benutzer die Suchergebnisse bewertet hat, wird diese Bewertung beim Start der nächsten Suche an den Suchalgorithmus zurückgemeldet.
\item Der Benutzer kann einen neuen Suchvorgang starten, ohne die Ergebnisse zu bewerten.
\item Der Benutzer kann durch die zurück-Funktion die letzten Schritte rückgängig machen.
\item Das Programm kann jederzeit durch den Benutzer beendet werden, insbesondere während die Suche läuft.
% Schnittstellen Daten und Suchverfahren
\item Es gibt eine Schnittstelle für die Anwendung der gegebenen Suchverfahren.
\item Es existiert eine einheitliche Schnittstelle für die Bild- und Videodateien.
\item Das Programm unterstützt mehrere Bildformate.
\item Es kann ein Standardordner für die Datensatzbibliothek per Kommandozeile übergeben werden.
%technisch
\item Das Programm läuft unter Microsoft Windows 8 und Linux (Ubuntu 14.04 und Fedora 22).
\end{enumerate}
\subsection{Wunschkriterien}
\begin{enumerate} [label=\bfseries /WK \arabic*0/, leftmargin=*]
\item Der Benutzer kann auch auf Englisch durch das Programm geführt werden.
\item Das Programm unterstützt auch Videodateien im Gegensatz zu Videos aus Einzelbildern.
\item Der Benutzer kann sich die Bilder der Datensatzauswahl und die Suchergebnisse als Vollbild anzeigen lassen.
\item Der Benutzer kann zwei unterschiedliche Suchverfahren nacheinander verwenden. Als Suchergebnis wird die Schnittmenge oder die Vereinigungsmenge angezeigt.
\item Der Benutzer kann den Suchvorgang stoppen.
\item Nachdem der Suchvorgang beendet ist, wird der Benutzer mit einem Ton benachrichtigt. Diesen kann er an- oder ausschalten.
\item Der Benutzer kann für die Suchergebnisse einen Wert zwischen 0 und 10 als \gls{Feedback} festlegen.
\item Der Benutzer kann das Suchergebnis als \gls{Lesezeichen} speichern.
\item Die letzten Suchergebnisse werden in der \gls{Chronik} gespeichert.
\item Das Programm bietet eine Vollbildmodus für Präsentationen an.
\item Die Historie der zuletzt verwendeten Datensätze wird erinnert.
\end{enumerate}
\subsection{Abgrenzungskriterien}
\begin{enumerate} [label=\bfseries /AK \arabic*0/, leftmargin=*]
\item Das Programm muss nicht unter Microsoft Windows XP und Apple OS X laufen. 
\item Es ist nicht für Tablets und Handys vorgesehen.
\item Zur Laufzeit können keine Suchverfahren hinzugefügt werden.
\end{enumerate}
\pagebreak


\section{Produkteinsatz}
\subsection{Andwendungsbereiche}
\subsection{Zielgruppen}
\subsection{Betriebsbedingungen}


\section{Produktumgebung}
\subsection{Software}
\begin{itemize}
\item Die Software l\"auft unter Ubuntu 14 / Windows 8
\end{itemize}
\subsection{Hardware}
\begin{itemize}
\item Es wird ein farbiges Display, auf dem Bilder und Videos angezeigt werden können, benötigt. 
\item Die Mindestauflösung des Displays muss 1024 x 768 Pixel betragen.
\end{itemize}


\section{Funktionale Anforderungen}
% !TeX spellcheck = de_DE_frami
\subsection{Pflichtanforderungen}
\begin{description}
	\item[\req{F 10}] Starten des Programms
	\item[\req{F 20}] Beenden des Programms
	\item[\req{F 30}] Erkennen verfügbarer Suchverfahren
	\item[\req{F 40}] Anzeigen eines Hilfe-Dialogs mit Hinweisen zur Benutzung des Programms
	\item[\req{F 50}] Anzeigen eines About-Dialogs mit Informationen zum Programm
	\newline
%	\item[\req{F 0}] Er\"offnen eines neuen Suchkontexts
%	\item[\req{F 0}] Er\"offnen eines neuen \glslink{Suchkontext}{Suchkontexts}
	\item
	\item[\req{F 60}] Wiederanzeigen der Ergebnisse einer vorherigen Suche (siehe auch \req{F 205})
	\item[\req{F 70}] Anzeigen einer Bibliothek von Datensätzen
	\item[\req{F 71}] Hinzufügen von Datensätzen zur Bibliothek
	\item[\req{F 72}] Ausw\"ahlen eines Datensatzes aus der Bibliothek
	\item[\req{F 80}] Anzeigen einer Übersicht der Bilder bzw. Videos des gewählten Datensatzes
	\item[\req{F 85}] Anzeigen einer größeren Darstellung eines ausgewählten Bildes bzw. Videos
	\item[\req{F 86}] Zoomen und Scrollen in der größeren Darstellung
	\item[\req{F 90}] Ausw\"ahlen eines Bildes bzw. Videos aus den gewählten Datensätzen als Suchvorlage
	\item[\req{F 100}] Anzeigen annotierter Bildbereiche
	\item[\req{F 105}] Auswahl eines annotierten Bildbereiches als Suchvorlage
	\item[\req{F 110}] Einschränkung der Suchvorlage auf einen benutzerdefinierten rechteckigen Bereich (falls keine Annotation gewählt wurde)
	\item[\req{F 120}] Auswahl eines für ggf. gewählte Annotation geeigneten Suchverfahrens
	\item[\req{F 125}] Festlegen der für das gewählte Suchverfahren spezifischen Parameter
	\item[\req{F 128}] Anzeigen der zurzeit eingestellten Parameter zur Überprüfung durch den Benutzer
	\newline
	\item[\req{F 130}] Starten der Suche
	\item[\req{F 140}] Anzeigen einer Fortschrittsanimation während des Suchvorgangs
	\item[\req{F 150}] Abbrechen der Suche
	\newline
	\item[\req{F 160}] Anzeigen einer \"Ubersicht der Suchergebnisse
	\item[\req{F 170}] Anzeigen eines einzelnen Suchergebnisses
	\item[\req{F 180}] Abspielen eines Videos aus Einzelbildern
	\item[\req{F 190}] Einstellen eines Feedbacks zu einem Suchergebnis
	\item[\req{F 200}] \"Ubermitteln des Feedbacks an das Suchverfahren
	\item[\req{F 205}] Speichern der Suchergebnisse (siehe auch \req{F 60})
	\newline
	\item[\req{F 210}] Unterstützung der Bildformate JPEG, PNG und BMP
	\item[\req{F 220}] Unterstützung von Videos in Form von Einzelbildern in untert\"utzten Bildformaten
\end{description}

\subsection{Wunschanforderungen}
\begin{description}
	\item[\req{FW 10}] nicht-flüchtige Auswahl der Übersetzungen der GUI in Deutsch und Englisch
	\newline
	\item[\req{FW 20}] Wählen einer Suchvorlage, die nicht in den gewählten Datensätzen enthalten ist (erweitert \req{F 90})
	\item[\req{FW 25}] Wählen von mehreren Datensätzen in denen gesucht wird (erweitert \req{F 72})
	\newline
	\item[\req{FW 30}] Abspielen eines Benachrichtigungstons bei Abschluss einer Suche
	\item[\req{FW 40}] Ein-/Abschalten des Benachrichtigungstons
	\item[\req{FW 50}] Abspielen von Videos (mit Ausgabe der Audiospur)
	\item[\req{FW 60}] Durchführen einer weiteren Suche in den Ergebnissen einer bereits beendeten Suche
	\newline
	\item[\req{F 70}] Wechsel in einen Vollbildmodus für Präsentation
%	\newline
%	\item[\req{F 0}] Duplizieren des aktuellen Suchkontextes
%	\item[\req{F 0}] Wechsel des aktuellen Suchkontextes %unklar
%	\item[\req{F 0}] nicht-fl\"uchtiges Speichern des Suchkontexts
%	\item[\req{F 0}] Laden eines gespeicherten Suchkontextes
%	\item[\req{F 0}] Verkn\"upfen (bilden von Vereinigungs- oder Schnittmenge) von Suchergebnissen verschiedener Suchkontexte
\end{description}


\section{Nichtfunktionale Anforderungen}
\begin{enumerate} [label=\bfseries /NF \arabic*0/, leftmargin=*]
  \item Die Anwendung soll möglichst intuitiv benutzbar sein.
  \item Das Programm soll während langer Vorgänge ein visuelles Feedback übermitteln.
  \item Die Videos müssen mit einer Auflösung  von 720p bei 25 Frames pro Sekunde flüssig laufen.
  \item Bei Fehleingaben oder größeren Dateien darf die Anwendung nicht abstürzen, sondern muss sie richtig behandeln können, bzw. aussagekräftige Fehlermeldungen anzeigen.
  \item Die Anwendung muss gut erweiterbar sein, was durch den Einsatz von \glslink{Entwurfsmuster}{Entwurfsmustern} gewährleistet werden soll.
  \item Die Datensätze dürfen von der Anwendung nicht geändert werden.
  \item Eine hohe Portierbarkeit soll durch den Einsatz von \gls{Qt} und der Vermeidung von Spezialbibliotheken gewährleistet werden.
  \item Ausgetauschte und hinzugefügte Datensätze und \gls{Suchverfahren} sollen ohne Neukompilierung nach Programmneustart zur Verfügung stehen.
  \item Die Oberfläche soll nie einfrieren, was durch einen separaten Thread gewährleistet wird.
\end{enumerate}
\pagebreak


\section{Produktdaten}
\begin{itemize}
	\item Bilder und Videos inklusive Annotationen bilden den Input des Programms.
	\item Die Einstellungsdatei speichert die vom Benutzer gewählte Sprache. Falls \req{FW 40} umgesetzt wird, muss auch die Einstellung zum Signalton in der Einstellungsdatei gespeichert werden.
	\item Falls \req{WK 10} umgesetzt wird, müssen die Übersetzungen gespeichert werden.
	\item Es wird eine Suchhistorie gespeichert, die letzten Suchergebnisse sind somit über mehrere Programmstarts abrufbar. 
	\item Eine Historie der zuletzt verwendeten Datensätze wird gespeichert.
	\item Lesezeichen, die durch den Benutzer erstellt werden, verweisen auf spezielle Suchergebnisse.
	\item Die Beschreibung des Ziels und der Aufgaben des Browsers ist zu speichern.
	\item Das Programm greift auf Suchverfahren und dynamisch gebundene Bibliotheken zu.
	\item Falls das \req{WK 110} umgesetzt wird, sollen die Vorschaubildchen gespeichert werden.
\end{itemize}
\pagebreak


\section{Globale Testf\"alle}
% !TeX spellcheck = de_DE_frami
Zum Testen während der Entwicklung und zum Durchspielen der Testszenarien werden mindestens nachfolgende Testdaten benötigt:
\begin{itemize}
	\item ein Bilddatensatz (mit mehreren Bildern und mindestens einem annotiertem Bild)
	\item ein Datensatz mit mindestens einem Video aus Einzelbildern
	\item ein Suchverfahren, das eine beliebige Sortierung als Suchergebnis zurück gibt
	\item ein weiteres Testsuchverfahren, dass eine alphabetische Sortierung als Suchergebnis zurück % alphabetisch worin?
	\item ein Suchverfahren, dass auf eine Annotation des oben geforderten annotierten Bildes anwendbar ist
\end{itemize}
Insbesondere die Schnittstelle zwischen Suchverfahren und Anwendung soll durch diese Testdaten überprüft werden.

Die folgenden Testszenarien beschreiben den Arbeitsablauf des Benutzers im Detail.

\begin{description} %[label=\bfseries /TS \arabic*0/]
	\item[\req{TF 10}] Grundfunktionalität % Vorgang zu generisch beschrieben; ToDo: bestimmte Datensätze und Verfahren zur konkretisierung spezifizieren und hier verwenden
	\begin{enumerate}
		\item
		\begin{description}
			\item[Vorbedingung] Programm ist startbereit; es wurden weder Datensätze noch Suchverfahren entfernt oder manipuliert
			\item[Aktion] Benutzer startet Programm (\req{F 10})
			\item[Nachbedingung] die Startansicht der GUI wird angezeigt; die Bibliothek wird angezeigt (\req{F 80}) und enthält mindestens einen Datensatz
		\end{description}
		\item
		\begin{description}
			\item[Aktion] Der Benutzer doppelklickt einen Datensatz (\req{F 100})
			\item[Nachbedingung] Es wird eine Übersicht des Datensatzes angezeigt (\req{F 110})
		\end{description}
		\item
		\begin{description}
			\item[Aktion] Der Benutzer klickt auf ein Bild aus der Übersicht (\req{F 140})
			\item[Nachbedingung] Das Bild wird groß dargestellt (\req{F 120})
		\end{description}
		\item
		\begin{description}
			\item[Aktion] Der Benutzer rechtsklickt auf einen Bereich des Bildes der nicht annotiert ist
			\item[Nachbedingung] Es wird ein Kontextmenü angezeigt in dem mindestens ein Suchverfahren zur Auswahl bereitsteht (\req{F 30})
		\end{description}
		\item
		\begin{description}
			\item[Aktion] Der Benutzer fährt mit der Maus über ein Suchverfahren
			\item[Nachbedingung] Es wird eine Beschreibung des Suchverfahrens angezeigt (\req{F 190})
		\end{description}
		\item
		\begin{description}
			\item[Aktion] Der Benutzer linksklickt auf eines der Suchverfahren (\req{F 180})
			\item[Nachbedingung] Das angezeigte Fenster enthält die Parameter des Suchverfahrens und entsprechende UI-Elemente sie festzulegen
		\end{description}
		\item
		\begin{description}
			\item[Aktion] Der Benutzer nimmt seine Eingaben vor und bestätigt sie (\req{F 200})
			\item[Nachbedingung] Es werden die vorgenommenen Einstellungen angezeigt (\req{F 210}); eine Schaltfläche zum Starten der Suche wird angezeigt
		\end{description}
		\item
		\begin{description}
			\item[Aktion] Der Benutzer startet die Suche (\req{F 230})
			\item[Nachbedingung] Eine Animation zeigt die Aktivität des Suchverfahrens an (\req{F 240})
		\end{description}
		\item
		\begin{description}
			\item[Aktion] Der Benutzer wartet
			\item[Nachbedingung] Eine Übersicht über die Ergebnisse der Suche wird angezeigt (\req{F 260})
		\end{description}
		\item
		\begin{description}
			\item[Aktion] Der Benutzer linksklickt auf ein Ergebnis
			\item[Nachbedingung] Das Ergebnis wird groß dargestellt (\req{F 270})
		\end{description}
		\item
		\begin{description}
			\item[Aktion] Der Benutzer wählt die Schaltfläche zur Rückkehr
			\item[Nachbedingung] Es werden wieder die vorgenommenen Einstellungen angezeigt (\req{F 210}); eine Schaltfläche zum Starten der Suche wird angezeigt
		\end{description}
		\item
		\begin{description}
			\item[Aktion] Der Benutzer Startet die Suche erneut
			\item[Nachbedingung] Eine Animation zeigt die Aktivität des Suchverfahrens an (\req{F 240})
		\end{description}
		\item
		\begin{description}
			\item[Aktion] Der Benutzer bricht die Suche ab
			\item[Nachbedingung] Es werden wieder die vorgenommenen Einstellungen angezeigt (\req{F 210}); eine Schaltfläche zum Starten der Suche wird angezeigt
		\end{description}
		\item
		\begin{description}
			\item[Aktion] Der Benutzer beendet das Programm über den üblichen Schließen-Button (\req{F 20})
			\item[Nachbedingung] die GUI ist nicht mehr sichtbar; der Prozess hat terminiert
		\end{description}
	\end{enumerate}

	\item[\req{TF 20}] Feedback und Lesezeichen
	\begin{enumerate}
		\item
		\begin{description}
			\item[Vorbedingung] Es wurde gerade eine Suche durchgeführt; es werden die Ergebnisse der Suche angezeigt
			\item[Aktion] Der Benutzer stellt ein positives und ein negatives Feedback für je ein Suchergebnis ein (\req{FW 290})
			\item[Nachbedingung] Das gegebene Feedback wird in der Übersicht dargestellt
		\end{description}
		\item
		\begin{description}
			\item[Aktion] Der Benutzer klickt auf die Schaltfläche zum Speichern der Suchergebnisse (\req{F 310}), gibt einen einen Namen für das Lesezeichen ein und beendet anschließend das Programm (\req{F 20})
			\item[Nachbedingung] Das Programm wurde beendet
		\end{description}
		\item
		\begin{description}
			\item[Aktion] Der Benutzer startet das Programm erneut (\req{FW 10})
			\item[Nachbedingung] Die Startansicht der GUI wird angezeigt
		\end{description}
		\item
		\begin{description}
			\item[Aktion] Der Benutzer klickt auf die Schaltfläche für das Anzeigen der Lesezeichen (\req{F 70})
			\item[Nachbedingung] in der angezeigten Liste ist mindestens das zuvor angelegte Lesezeichen zu finden
		\end{description}
		\item
		\begin{description}
			\item[Aktion] Der Benutzer wählt das zuvor angelegte Lesezeichen (\req{F 60}, \req{F 310})
			\item[Nachbedingung] Die bereits zuvor angezeigten Suchergebnisse werden in der selben Reihenfolge wieder angezeigt; das zuvor festgelegte Feedback wird wieder angezeigt
		\end{description}
	\end{enumerate}

	\item[\req{TF 30}] erweiterte Funktionen zu Bilddatensätzen
	\begin{enumerate}
		\item
		\begin{description}
			\item[Vorbedingung] Es wurde ein Datensatz (bestehend aus Bildern) gewählt; das Programm zeigt die Übersicht der Bilder des Datensatzes; es ist ein Foto eines realen Motivs im Datensatz vorhanden; das Foto weist annotierte Bildbereiche auf
			\item[Aktion] Der Benutzer wählt dieses Bild zur vergrößerten Ansicht (\req{F 120})
			\item[Nachbedingung] das Bild wird in vergrößerter Ansicht angezeigt; es werden die annotierten Bildbereiche angezeigt (\req{F 150})
		\end{description}
		\item
		\begin{description}
			\item[Aktion] Der Benutzer verwendet die Schaltflächen, um zum nächsten und vorherigen Bild zu wechseln (\req{F 135})
			\item[Nachbedingung] Das Bild in der größeren Ansicht wechselt vor bzw. zurück entsprechend der durch die Übersicht gegebenen Reihenfolge
		\end{description}
		\item
		\begin{description}
			\item[Aktion] Der Benutzer kehrt zum annotierten Bild zurück; der Benutzer rechtsklickt auf einen annotierten Bildbereich (\req{F 160})
			\item[Nachbedingung] Es wird ein Kontextmenü angezeigt in dem mindestens ein Suchverfahren zur Auswahl bereitsteht (\req{F 30})
		\end{description}
		\item
		\begin{description}
			\item[Aktion] Der Benutzer linksklickt auf ein Suchverfahren (\req{F 180})
			\item[Nachbedingung] Es wird der gewählte annotierte Bildbereich angezeigt (\req{F 210}); eine Schaltfläche zur Rückkehr zur Übersicht des Datensatzes wird angezeigt
		\end{description}
		\item
		\begin{description}
			\item[Aktion] Der Benutzer linksklickt die Schaltfläche zur Rückkehr (\req{F 220})
			\item[Nachbedingung] Das eben gewählte Bild wird in der vergrößerten Ansicht dargestellt
		\end{description}
		\item
		\begin{description}
			\item[Aktion] Der Benutzer wählt das Werkzeug zur Auswahl eines Bildbereiches und wählt per Drag-and-Drop in der vergrößerten Ansicht einen Bildbereich (\req{F 170})
			\item[Nachbedingung] Der gewählte Bereich wird dargestellt
		\end{description}
		\item
		\begin{description}
			\item[Aktion] Der Benutzer rechtsklickt in den gewählten Bereich und linksklickt auf ein Suchverfahren (\req{F 180})
			\item[Nachbedingung] Es wird der gewählte Bildbereich angezeigt (\req{F 210}); eine Schaltfläche zur Rückkehr zur Übersicht des Datensatzes wird angezeigt
		\end{description}
		\item
		\begin{description}
			\item[Aktion] Der Benutzer linksklickt die Schaltfläche zur Rückkehr (\req{F 220})
			\item[Nachbedingung] Das eben gewählte Bild wird in der vergrößerten Ansicht dargestellt; der gewählte Bildbereich wird angezeigt
		\end{description}
		\item
		\begin{description}
			\item[Aktion] Der Benutzer verwendet die Schaltflächen, um zum nächsten und vorherigen Bild zu wechseln (\req{F 135})
			\item[Nachbedingung] Das Bild in der größeren Ansicht wechselt vor bzw. zurück entsprechend der durch die Übersicht gegebenen Reihenfolge
		\end{description}
	\end{enumerate}

	\item[\req{TF 40}] Datensatz zur Bibliothek hinzufügen
	\begin{enumerate}
		\item
		\begin{description}
			\item[Vorbedingung] Programm ist startbereit; ein korrekter Datensatz liegt auf der Festplatte; % "korrekt" = ?
			\item[Aktion] Benutzer startet Programm (\req{F 10})
			\item[Nachbedingung] Startbildschirm der GUI wird angezeigt; die Bibliothek wird angezeigt (\req{F 80})
		\end{description}
		\item
		\begin{description}
			\item[Aktion] Der Benutzer klickt die Schaltfläche zum Hinzufügen eines neuen Datensatzes (\req{F 100})
			\item[Nachbedingung] Ein Dialog der die Auswahl eines Datensatzes von der Festplatte ermöglicht wird angezeigt
		\end{description}
		\item
		\begin{description}
			\item[Aktion] Der Benutzer wählt einen korrekten Datensatz von der Festplatte und bestätigt seine Wahl (\req{F 90})
			\item[Nachbedingung] der Dialog wird nicht mehr angezeigt; die Liste der gewählten Datensätze enthält den eben gewählten Datensatz
		\end{description}
	\end{enumerate}

	\item[\req{TF 60}] Videos aus Einzelbildern anzeigen
	\begin{enumerate}
		\item
		\begin{description}
			\item[Vorbedingung] es wurde ein Datensatz gewählt, der mindestens ein Video aus Einzelbildern enthält.
			\item[Aktion] der Benutzer wählt eines der Videos zur Ansicht (\req{F 120})
			\item[Nachbedingung] ein Videoplayer wird angezeigt; eine Schaltfläche zum Starten des Videos ist verfügbar
		\end{description}
		\item
		\begin{description}
			\item[Aktion] der Benutzer klickt auf die Schaltfläche zum Starten des Videos (\req{F 280})
			\item[Nachbedingung] das Video wird abgespielt
		\end{description}
	\end{enumerate}

	\item[\req{TF 50}] Sprachwahl
	\begin{enumerate}
		\item
		\begin{description}
			\item[Vorbedingung] Programm ist startbereit; Einstellungen wurden noch nicht geändert
			\item[Aktion] Benutzer startet Programm (\req{F 10})
			\item[Nachbedingung] deutsches GUI wird angezeigt; Menüpunkt 'Sprache' ist sichtbar, es sind die Sprachen Deutsch und Englisch wählbar; Deutsch ist als aktuell gewählte Sprache markiert
		\end{description}
		\item
		\begin{description}
			\item[Aktion] Der Benutzer wählt über den Menüpunkt 'Sprache' die englische Sprache aus (\req{FW 10})
			\item[Nachbedingung] die GUI wird auf englisch dargestellt
		\end{description}
		\item
		\begin{description}
			\item[Aktion] Der Benutzer beendet das Programm über den üblichen Schließen-Button (\req{F 20})
			\item[Nachbedingung] die GUI ist nicht mehr sichtbar; der Prozess hat terminiert
		\end{description}
		\item
		\begin{description}
			\item[Aktion] Der Benutzer startet das Programm (\req{F 10})
			\item[Nachbedingung] GUI wird auf Englisch angezeigt (\req{FW 10}); unter dem Menüpunkt 'Sprache' ist nun Englisch als aktuell gewählte Sprache markiert
		\end{description}
	\end{enumerate}

	\item[\req{TF 60}] weitere Menüpunkte
	\begin{enumerate}
		\item
		\begin{description}
			\item[Vorbedingung] das Menü ist verfügbar; deutsch ist die gewählte Sprache (falls \req{FW 10} implementiert)
			\item[Aktion] der Benutzer wählt über den Menüpunkt ganz rechts 'About' aus (\req{F 50})
			\item[Nachbedingung] ein Dialog mit Informationen zum Programm wird angezeigt
		\end{description}
		\item
		\begin{description}
			\item[Aktion] der Benutzer wählt über den Menüpunkt ganz rechts 'Hilfe' aus (\req{F 40})
			\item[Nachbedingung] ein Dialog mit Hinweisen zur Benutzung des Programms wird angezeigt
		\end{description}
	\end{enumerate}
\end{description}
\pagebreak


\section{Systemmodelle}
%\subsection{Szenarien} % arguably useless
\subsection{Anwendungsf\"alle}
%\subsection{Objektmodelle} % belongs to design phase
%\subsection{Dynamische Modelle} % belongs to design phase
\subsection{Benutzerschnittstelle}


\section{Benutzerschnittstelle}
\includegraphics[width=1\linewidth]{img/MainWindow}
Das erste Fenster beinhaltet die Bibliothek und außerdem ein Menü. In der Menüleiste kann im Punkt Sprache die aktuelle Sprache ausgewählt werden. Unter Lesezeichen können Suchergebnisse geöffnet werden, die als Lesezeichen gespeichert wurden. In der Chronik kann ein Suchergebnis der letzten 5 Suchen geöffnet werden. Über den Menüpunkt Hilfe kann eine Anleitung oder ein About-Fenster geöffnet werden.
In der Bibliothek befinden sich die Datensätze aus dem Standardordner und zuletzt verwendete. Daraus kann ein Datensatz ausgewählt werden, in dem sich das Bild/Video befindet, nach dem gesucht werden soll. Mithilfe des Buttons 'neuer Datensatz' kann auch ein anderer Ordner als Datensatz gewählt werden.

\includegraphics[width=1\linewidth]{img/FileChooser}
Hier kann der Benutzer einen Bild- oder Videodatensatz auswählen.

\includegraphics[width=1\linewidth]{img/Fotoanzeiger}
Über den 'Home' Button oben links kommt der Benutzer zurück zur Bibliothek. Beim Klicken des 'zurück' Buttons kann ein Schritt rückgängig gemacht werden, man kommt also zum vorherigen Fenster. Diese Buttons werden in den nächsten Fenstern immer an der gleichen Stelle sein.

Nach der Auswahl eines Datensatzes öffnet sich ein Bild-/Videoanzeiger, je nach dem, ob ein Bild- oder Videodatensatz gewählt wurde. Über die Buttons 'vorheriges' und 'nächstes' kann der Benutzer durch die Bilder/Videos browsen und ein geeignetes auswählen. Über den Button Vollbild kann das Bild/Video im Vollbildmodus angezeigt werden.
Bei einem Rechtsklick auf das Bild werden die Algorithmen aufgelistet, die für diese Suche zur Verfügung stehen. Fährt man mit der Maus über einen solchen Algorithmus, dann erscheint eine Beschreibung zu diesem.
In den Bildern werden außerdem, falls vorhanden, Annotationen angezeigt. Davon kann eine (mit Klick darauf) ausgewählt werden. Optional kann auch ein eigenes Rechteck gezogen werden. Mit Rechtsklick auf eine Annotation erscheint eine Auswahl aller Suchalgorithmen, die für diese Annotation in Frage kommen. Der Benutzer kann durch Klicken auf einen Algorithmusnamen mit dem Programm fortfahren.

\includegraphics[width=1\linewidth]{img/Videoanzeiger}
Der Videoanzeiger bietet alle Funktionen des oben beschriebenen Fotoanzeigers. Zusätzlich können auch Videos abgespielt werden. Dabei wird im Player, ähnlich zu YouTube, Play, Pause, ein Fortschrittsbalken und die vergangene Zeit angezeigt. 


Nach der Auswahl eines Suchverfahrens, kann man Parameter für den Suchalgorithmus festlegen. Außerdem können weitere Datensätze ausgewählt werden, in denen gesucht werden soll.

\includegraphics[width=1\linewidth]{img/Ueberpruefung}
Im Überprüfungsfenster wird noch einmal die aktuelle Auswahl angezeigt: der gewählte Bild-/Videoausschnitt, die Datensätze, in denen gesucht wird, der Suchalgorithmus und die Parameter. Wenn der Benutzer an diesen Werten nichts mehr ändern will, kann er nun über den Button 'Suche starten' das Suchverfahren mit der gezeigten Auswahl starten.

Während der Suche wird eine Fortschrittsanimation in dem Fenster angezeigt, in dem dann die Suchergebnisse zu sehen sein werden. Wenn alle Ergebnisse angezeigt werden, kann das Suchergebnis über den Button als Lesezeichen gespeichert werden.\newline 
Der Benutzer kann durch einen Klick auf ein Bild dieses als positiv (grüner Kasten) bewerten. Durch einen weiteren Klick wird es negativ (roter Kasten) und durch noch einen Klick wieder neutral (kein Kasten) bewertet. Durch einen Klick auf den Button 'Erneut suchen', wird das Feedback an den Algorithmus übermittelt und eine neue verbesserte Suche gestartet.


\section{Entwicklungsumgebung}
\subsection{Software}
\begin{itemize}
\item Betriebssysteme
	\begin{itemize}[label={--}]
		\item Windows 8
		\item Ubuntu 14
	\end{itemize}
\item Entwicklung
	\begin{itemize}[label={--}]
		\item C++11 mit Standardbibliotheken
		\item Qt5.5 Bibliotheken
		\item QtDesigner
		\item QtCreator
	\end{itemize}
\item Versionsverwaltung
	\begin{itemize}[label={--}]
		\item Git 1.9.1
	\end{itemize}
\item Dokumentation
	\begin{itemize}[label={--}]
		\item LaTeX
	\end{itemize}
\end{itemize}

\subsection{Hardware}
\begin{itemize}
	\item Diverse Standardrechner (Intel/CPU CPUs)
\end{itemize}


\section{Glossar}
%\printglossaries

\end{document}
