\subsection{Software}
\begin{itemize}
\item Sprache im Team
	\begin{itemize}[label={--}]
		\item Es wird auf Deutsch kommuniziert.
	\end{itemize}
\item Betriebssystem
	\begin{itemize}[label={--}]
		\item Die zum Entwickeln verwendeten Betriebssysteme sind Windows 7 und 8, Fedora 22 und Ubuntu 14.
	\end{itemize}
\item Entwicklung
	\begin{itemize}[label={--}]
		\item Es wird mit C++14 und der C++14-Standardbibliothek programmiert
		\item Es wird mit der \gls{Qt} 5.5 Bibliothek gearbeitet. Als Tools werden \gls{Qt Designer} und \gls{Qt Creator} verwendet.
	\end{itemize}
\item \gls{Versionsverwaltung}
	\begin{itemize}[label={--}]
		\item Die Versionierung erfolgt mittels \gls{Git} 1.9.1.
	\end{itemize}
\item Dokumentation
	\begin{itemize}[label={--}]
		\item Zum Aufbau von \glslink{Phasendokument}{Phasendokumenten} wird hauptsächlich LaTeX verwendet.
	\end{itemize}
\item Quelltext
	\begin{itemize}[label={--}]
		\item Quellcode und Kommentare werden auf Englisch verfasst.
		\item Es werden \enquote{C++ coding style conventions} verwendet.
		\item Die Dokumentation des Quelltextes erfolgt über \gls{Doxygen} auf Englisch.
	\end{itemize}
\item Qualitätssicherung
	\begin{itemize}[label={--}]
		\item \gls{Qt Test} wird für automatisierte Tests von einzelnen Komponenten der Anwendung verwendet.
	\end{itemize}

\end{itemize}

\subsection{Hardware}
\begin{itemize}
	\item Es werden diverse Standardrechner mit Intel und AMD CPUs zum Aufbau des Programms benutzt.
\end{itemize}
\pagebreak
