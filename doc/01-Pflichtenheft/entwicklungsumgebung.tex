\subsection{Software}
\begin{itemize}
\item Sprache im Team
	\begin{itemize}[label={--}]
		\item Es wird hauptsächlich auf Deutsch kommuniziert.
	\end{itemize}
\item Betriebssystem
	\begin{itemize}[label={--}]
		\item Betriebssysteme, die von den Mitgliedern des Prjekts zum Enwickeln verwendet wurden, sind Windows 8, Fedora 22 und Ubuntu 14.
	\end{itemize}
\item Entwicklung
	\begin{itemize}[label={--}]
		\item Es wird auf C++ (Version 14) programmiert, wobei nur Standardbibliotheken in Betracht kommen.
		\item Es wird mit Qt5.5 Bibliotheken gearbeitet, wobei als grundsätzliche Tools zum Entwickeln  QtDesigner und QtCreator verwendet werden.
	\end{itemize}
\item Versionsverwaltung
	\begin{itemize}[label={--}]
		\item Die Versionierung erfolgt mittels Git 1.9.1.
	\end{itemize}
\item Dokumentation
	\begin{itemize}[label={--}]
		\item Zum Aufbau von Phasendokumenten wird hauptsächlich LaTeX verwendet.
	\end{itemize}
\item Quelltext
	\begin{itemize}[label={--}]
		\item Der Code des Programms sowie die Kommentare werden auf Englisch verfasst.
		\item Es werden "C++ coding style conventions" verwendet.
		\item Die Dokumentation des Quelltextes erfolgt über Doxygen.
	\end{itemize}
\end{itemize}

\subsection{Hardware}
\begin{itemize}
	\item Es werden diverse Standardrechner mit Intel und AMD CPUs zum Aufbau des Programms benutzt.
\end{itemize}
