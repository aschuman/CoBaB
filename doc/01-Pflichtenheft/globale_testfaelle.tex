% !TeX spellcheck = de_DE_frami
Zum Testen während der Entwicklung und zum Durchspielen der Testszenarien werden mindestens nachfolgende Testdaten benötigt:
\begin{itemize}
	\item ein Bilddatensatz (mit mehreren Bildern und mindestens einem annotiertem Bild)
	\item ein Datensatz mit mindestens einem Video aus Einzelbildern
	\item ein Testsuchverfahren, das eine beliebige Sortierung als Suchergebnis zurück gibt
	\item ein Suchverfahren, dass auf eine Annotation des oben geforderten annotierten Bildes anwendbar ist
\end{itemize}
Insbesondere die Schnittstelle zwischen Suchverfahren und Anwendung soll durch diese Testdaten überprüft werden.

Die folgenden Testszenarien beschreiben den Arbeitsablauf des Benutzers im Detail.

\subsection{Testszenarien der Pflichtanforderungen}
\begin{enumerate} [label=\bfseries /TS \arabic*0/, leftmargin=*]
	\item Grundfunktionalität % Vorgang zu generisch beschrieben; ToDo: bestimmte Datensätze und Verfahren zur konkretisierung spezifizieren und hier verwenden
	\begin{enumerate}[leftmargin=0pt, itemindent=0pt]
		\item
		\begin{description}
			\item[Vorbedingung] Das Programm ist startbereit. Es wurden weder Datensätze noch Suchverfahren entfernt oder manipuliert.
			\item[Aktion] Der Benutzer startet das Programm. (\ref{f:programmstart})
			\item[Nachbedingung] Die Startansicht der GUI wird angezeigt. Die Bibliothek wird angezeigt (\ref{f:bibliothek_anzeigen}) und enthält alle Datensätze des Standardordners
		\end{description}
		\item
		\begin{description}
			\item[Aktion] Der Benutzer doppelklickt einen Datensatz (\ref{f:auswahl_eines_datensatzes})
			\item[Nachbedingung] Es wird eine Übersicht des Datensatzes angezeigt (\ref{f:uebersicht_anzeigen})
		\end{description}
		\item
		\begin{description}
			\item[Aktion] Der Benutzer klickt auf ein Bild aus der Übersicht (\ref{f:bildauswahl})
			\item[Nachbedingung] Das Bild wird groß dargestellt (\ref{f:anzeigen_groessere_darstellung})
		\end{description}
		\item
		\begin{description}
			\item[Aktion] Der Benutzer rechtsklickt auf einen Bereich des Bildes der nicht annotiert ist
			\item[Nachbedingung] Es wird ein Kontextmenü angezeigt in dem mindestens ein Suchverfahren zur Auswahl bereitsteht (\ref{f:erkennen_suchverfahren})
		\end{description}
		\item
		\begin{description}
			\item[Aktion] Der Benutzer fährt mit der Maus über ein Suchverfahren
			\item[Nachbedingung] Es wird eine Beschreibung des Suchverfahrens angezeigt (\ref{f:beschreibung_suchverfahren})
		\end{description}
		\item
		\begin{description}
			\item[Aktion] Der Benutzer linksklickt auf eines der Suchverfahren (\ref{f:auswahl_suchverfahren})
			\item[Nachbedingung] Das angezeigte Fenster enthält die Parameter des Suchverfahrens und entsprechende UI-Elemente, die ermöglichen sie festzulegen.
		\end{description}
		\item
		\begin{description}
			\item[Aktion] Der Benutzer nimmt seine Eingaben vor und bestätigt sie (\ref{f:parameterwahl})
			\item[Nachbedingung] Es werden die vorgenommenen Einstellungen angezeigt (\ref{f:ueberpruefung}); eine Schaltfläche zum Starten der Suche wird angezeigt
		\end{description}
		\item
		\begin{description}
			\item[Aktion] Der Benutzer startet die Suche (\ref{f:suche_starten})
			\item[Nachbedingung] Eine Animation zeigt die Aktivität des Suchverfahrens an (\ref{f:fortschrittsanimation})
		\end{description}
		\item
		\begin{description}
			\item[Aktion] Der Benutzer wartet
			\item[Nachbedingung] Eine Übersicht über die Ergebnisse der Suche wird angezeigt (\ref{f:suchergebnisse_anzeigen})
		\end{description}
		\item
		\begin{description}
			\item[Aktion] Der Benutzer linksklickt auf ein Ergebnis
			\item[Nachbedingung] Das Ergebnis wird groß dargestellt (\ref{fw:groesseres_suchergebnis})
		\end{description}
		\item
		\begin{description}
			\item[Aktion] Der Benutzer wählt die Schaltfläche zur Rückkehr
			\item[Nachbedingung] Es werden wieder die vorgenommenen Einstellungen angezeigt (\ref{f:ueberpruefung}); eine Schaltfläche zum Starten der Suche wird angezeigt
		\end{description}
		\item
		\begin{description}
			\item[Aktion] Der Benutzer startet die Suche erneut
			\item[Nachbedingung] Eine Animation zeigt die Aktivität des Suchverfahrens an (\ref{f:fortschrittsanimation})
		\end{description}
		\item
		\begin{description}
			\item[Aktion] Der Benutzer beendet das Programm über den üblichen Schließen-Button (\ref{f:beenden})
			\item[Nachbedingung] die GUI ist nicht mehr sichtbar; der Prozess hat terminiert
		\end{description}
	\end{enumerate}

	\item erweiterte Funktionen zu Bilddatensätzen
	\begin{enumerate}[leftmargin=0pt]
		\item
		\begin{description}
			\item[Vorbedingung] Es wurde ein Datensatz (bestehend aus Bildern) gewählt; das Programm zeigt die Übersicht der Bilder des Datensatzes; es ist ein Foto eines realen Motivs im Datensatz vorhanden; das Foto weist annotierte Bildbereiche auf
			\item[Aktion] Der Benutzer wählt dieses Bild zur vergrößerten Ansicht (\ref{f:anzeigen_groessere_darstellung})
			\item[Nachbedingung] das Bild wird in vergrößerter Ansicht angezeigt; es werden die annotierten Bildbereiche angezeigt (\ref{f:annotation_anzeigen})
		\end{description}
		\item
		\begin{description}
			\item[Aktion] Der Benutzer verwendet die Schaltflächen, um zum nächsten und vorherigen Bild zu wechseln (\ref{f:vorheriges_naechstes})
			\item[Nachbedingung] Das Bild in der größeren Ansicht wechselt vor bzw. zurück entsprechend der durch die Übersicht gegebenen Reihenfolge
		\end{description}
		\item
		\begin{description}
			\item[Aktion] Der Benutzer kehrt zum annotierten Bild zurück; der Benutzer rechtsklickt auf einen annotierten Bildbereich (\ref{f:annotation_auswaehlen})
			\item[Nachbedingung] Es wird ein Kontextmenü angezeigt in dem mindestens ein Suchverfahren zur Auswahl bereitsteht (\ref{f:erkennen_suchverfahren})
		\end{description}
		\item
		\begin{description}
			\item[Aktion] Der Benutzer linksklickt auf ein Suchverfahren (\ref{f:auswahl_suchverfahren})
			\item[Nachbedingung] Es wird die Parameterauswahl angezeigt
		\end{description}
		\item
		\begin{description}
			\item[Aktion] Der Benutzer wählt seine Parameter und bestätigt sie (\ref{f:parameterwahl})
			\item[Nachbedingung] Es wird der gewählte annotierte Bildbereich angezeigt (\ref{f:ueberpruefung}); eine Schaltfläche zur Rückkehr zur Parameterauswahl wird angezeigt
		\end{description}
		\item
		\begin{description}
			\item[Aktion] Der Benutzer linksklickt die Schaltfläche zur Rückkehr zweimal (\ref{f:einstellungen_nach_ueberpruefung_aendern})
			\item[Nachbedingung] Das eben gewählte Bild wird in der vergrößerten Ansicht dargestellt
		\end{description}
		\item
		\begin{description}
			\item[Aktion] Der Benutzer wählt das Werkzeug zur Auswahl eines Bildbereiches und wählt per Drag-and-Drop in der vergrößerten Ansicht einen Bildbereich (\ref{f:bereich_auswaehlen})
			\item[Nachbedingung] Der gewählte Bereich wird dargestellt
		\end{description}
		\item
		\begin{description}
			\item[Aktion] Der Benutzer rechtsklickt in den gewählten Bereich und linksklickt auf ein Suchverfahren (\ref{f:auswahl_suchverfahren})
			\item[Nachbedingung] Es wird die Parameterauswahl angezeigt
		\end{description}
		\item
		\begin{description}
			\item[Aktion] Der Benutzer wählt seine Parameter und bestätigt sie (\ref{f:parameterwahl})
			\item[Nachbedingung] Es wird der gewählte Bildbereich angezeigt (\ref{f:ueberpruefung}); eine Schaltfläche zur Rückkehr zur Parameterauswahl wird angezeigt
		\end{description}
		\item
		\begin{description}
			\item[Aktion] Der Benutzer linksklickt die Schaltfläche zur Rückkehr zweimal (\ref{f:einstellungen_nach_ueberpruefung_aendern})
			\item[Nachbedingung] Das eben gewählte Bild wird in der vergrößerten Ansicht dargestellt; der gewählte Bildbereich wird angezeigt
		\end{description}
		\item
		\begin{description}
			\item[Aktion] Der Benutzer verwendet die Schaltflächen, um zum nächsten und vorherigen Bild zu wechseln (\ref{f:vorheriges_naechstes})
			\item[Nachbedingung] Das Bild in der größeren Ansicht wechselt vor bzw. zurück entsprechend der durch die Übersicht gegebenen Reihenfolge
		\end{description}
	\end{enumerate}

	\item Datensatz zur Bibliothek hinzufügen
	\begin{enumerate}[leftmargin=0pt]
		\item
		\begin{description}
			\item[Vorbedingung] Programm ist startbereit; ein korrekter Datensatz liegt auf der Festplatte; % "korrekt" = ?
			\item[Aktion] Benutzer startet Programm (\ref{f:programmstart})
			\item[Nachbedingung] Startbildschirm der GUI wird angezeigt; die Bibliothek wird angezeigt (\ref{f:bibliothek_anzeigen})
		\end{description}
		\item
		\begin{description}
			\item[Aktion] Der Benutzer klickt die Schaltfläche zum Hinzufügen eines neuen Datensatzes (\ref{f:auswahl_eines_datensatzes})
			\item[Nachbedingung] Ein \gls{Dialog}, der die Auswahl eines Datensatzes von der Festplatte ermöglicht, wird angezeigt
		\end{description}
		\item
		\begin{description}
			\item[Aktion] Der Benutzer wählt einen korrekten Datensatz von der Festplatte und bestätigt seine Wahl (\ref{f:datensatz_hinzufuegen})
			\item[Nachbedingung] Der \gls{Dialog} wird nicht mehr angezeigt; es wird eine Übersicht des gewählten Datensatzes angezeigt (\ref{f:uebersicht_anzeigen})
		\end{description}
		\item
		\begin{description}
			\item[Aktion] Der Benutzer linksklickt die Schaltfläche zur Rückkehr
			\item[Nachbedingung] Die Liste der Bibliothek enthält den eben gewählten Datensatz
		\end{description}
	\end{enumerate}

	\item Videos aus Einzelbildern anzeigen
	\begin{enumerate}[leftmargin=0pt]
		\item
		\begin{description}
			\item[Vorbedingung] es wurde ein Datensatz gewählt, der mindestens ein Video aus Einzelbildern enthält.
			\item[Aktion] der Benutzer wählt eines der Videos zur Ansicht (\ref{f:anzeigen_groessere_darstellung})
			\item[Nachbedingung] ein Videoplayer wird angezeigt; eine Schaltfläche zum Starten des Videos ist verfügbar
		\end{description}
		\item
		\begin{description}
			\item[Aktion] der Benutzer klickt auf die Schaltfläche zum Starten des Videos (\ref{f:video_abspielen})
			\item[Nachbedingung] das Video wird abgespielt
		\end{description}
	\end{enumerate}

	\item weitere Menüpunkte
	\begin{enumerate}[leftmargin=0pt]
		\item
		\begin{description}
			\item[Vorbedingung] das Menü ist verfügbar
			\item[Aktion] der Benutzer wählt über den Menüpunkt ganz rechts \enquote{About} aus (\ref{f:about})
			\item[Nachbedingung] ein \gls{Dialog} mit Informationen zum Programm wird angezeigt
		\end{description}
		\item
		\begin{description}
			\item[Aktion] der Benutzer wählt über den Menüpunkt ganz rechts \enquote{Hilfe} aus (\ref{f:hilfe})
			\item[Nachbedingung] ein \gls{Dialog} mit Hinweisen zur Benutzung des Programms wird angezeigt
		\end{description}
	\end{enumerate}
\end{enumerate}

\subsection{Testszenarien der Wunschanforderungen}
\begin{enumerate} [label=\bfseries /TSW \arabic*0/, leftmargin=*]
	\item Feedback, Chronik und Lesezeichen
	\begin{enumerate}[leftmargin=0pt]
		\item
		\begin{description}
			\item[Vorbedingung] Es wurde gerade eine Suche durchgeführt; es werden die Ergebnisse der Suche angezeigt
			\item[Aktion] Der Benutzer stellt ein positives und ein negatives Feedback für je ein Suchergebnis ein (\ref{f:feedback})
			\item[Nachbedingung] Das gegebene Feedback wird in der Übersicht dargestellt
		\end{description}
		\item
		\begin{description}
			\item[Aktion] Der Benutzer beendet das Programm (\ref{f:beenden})
			\item[Nachbedingung] Das Programm wurde beendet
		\end{description}
		\item
		\begin{description}
			\item[Aktion] Der Benutzer startet das Programm erneut (\ref{f:programmstart})
			\item[Nachbedingung] Die Startansicht der GUI wird angezeigt
		\end{description}
		\item
		\begin{description}
			\item[Aktion] Der Benutzer wählt den Menüpunkt \enquote{Chronik} (\ref{fw:chronik_uebersicht})
			\item[Nachbedingung] Die zuletzt beendeten Suchen werden angezeigt
		\end{description}
		\item
		\begin{description}
			\item[Aktion] Der Benutzer wählt die zuletzt beendete Suche(\ref{fw:chronik}, \ref{fw:speichern_chronik})
			\item[Nachbedingung] Die bereits zuvor angezeigten Suchergebnisse werden in der selben Reihenfolge wieder angezeigt; das zuvor festgelegte Feedback wird wieder angezeigt
		\end{description}
		\item
		\begin{description}
			\item[Aktion] Der Benutzer klickt auf die Schaltfläche zum Speichern der Suchergebnisse (\ref{fw:speichern_lesezeichen}), gibt einen Namen für das Lesezeichen ein und beendet anschließend das Programm (\ref{f:beenden})
			\item[Nachbedingung] Das Programm wurde beendet
		\end{description}
		\item
		\begin{description}
			\item[Aktion] Der Benutzer startet das Programm erneut (\ref{f:programmstart})
			\item[Nachbedingung] Die Startansicht der GUI wird angezeigt
		\end{description}
		\item
		\begin{description}
			\item[Aktion] Der Benutzer klickt auf die Schaltfläche für das Anzeigen der Lesezeichen (\ref{fw:lesezeichen_uebersicht})
			\item[Nachbedingung] in der angezeigten Liste ist mindestens das zuvor angelegte Lesezeichen zu finden
		\end{description}
		\item
		\begin{description}
			\item[Aktion] Der Benutzer wählt das zuvor angelegte Lesezeichen (\ref{fw:lesezeichen}, \ref{fw:speichern_lesezeichen})
			\item[Nachbedingung] Die bereits zuvor angezeigten Suchergebnisse werden in der selben Reihenfolge wieder angezeigt; das zuvor festgelegte Feedback wird wieder angezeigt
		\end{description}
	\end{enumerate}

	\item Sprachwahl
	\begin{enumerate}[leftmargin=0pt]
		\item
		\begin{description}
			\item[Vorbedingung] Programm ist startbereit; Einstellungen wurden noch nicht geändert
			\item[Aktion] Benutzer startet Programm (\ref{f:programmstart})
			\item[Nachbedingung] deutsche GUI wird angezeigt; Menüpunkt \enquote{Sprache} ist sichtbar, es sind die Sprachen Deutsch und Englisch wählbar; Deutsch ist als aktuell gewählte Sprache markiert
		\end{description}
		\item
		\begin{description}
			\item[Aktion] Der Benutzer wählt über den Menüpunkt \enquote{Sprache} die englische Sprache aus (\ref{fw:sprache})
			\item[Nachbedingung] die GUI wird auf englisch dargestellt
		\end{description}
		\item
		\begin{description}
			\item[Aktion] Der Benutzer beendet das Programm über den üblichen Schließen-Button (\ref{f:beenden})
			\item[Nachbedingung] die GUI ist nicht mehr sichtbar; der Prozess hat terminiert
		\end{description}
		\item
		\begin{description}
			\item[Aktion] Der Benutzer startet das Programm (\ref{f:programmstart})
			\item[Nachbedingung] GUI wird auf Englisch angezeigt (\ref{fw:sprache}); unter dem Menüpunkt \enquote{Sprache} ist nun Englisch als aktuell gewählte Sprache markiert
		\end{description}
	\end{enumerate}

	\item Signalton
	\begin{enumerate}[leftmargin=0pt]
		\item
		\begin{description}
			\item[Vorbedingung] Das Programm befindet sich in der Bibliothek; Audioausgabe des Computers ist möglich und eingeschaltet
			\item[Aktion] Der Benutzer schaltet in den Einstellungen den Signalton ein, falls dieser noch nicht eingeschaltet war \ref{fw:signalton_einaus}
			\item[Nachbedingung] Die GUI zeigt an, dass der Signalton eingeschaltet ist
		\end{description}
		\item
		\begin{description}
			\item[Aktion] Der Benutzer beendet das Programm (\ref{f:beenden})
			\item[Nachbedingung] Das Programm ist beendet
		\end{description}
		\item
		\begin{description}
			\item[Aktion] Der Benutzer startet das Programm (\ref{f:programmstart})
			\item[Nachbedingung] Das Programm zeigt die Bibliothek an
		\end{description}
		\item
		\begin{description}
			\item[Aktion] Der Benutzer lässt sich die eben vorgenommenen Einstellungen bezüglich des Signaltons anzeigen
			\item[Nachbedingung] Die GUI zeigt an, dass der Signalton eingeschaltet ist
		\end{description}
		\item
		\begin{description}
			\item[Aktion] Der Benutzer wählt einen Datensatz, eine Suchvorlage, einen Algorithmus und die Parameter
			\item[Nachbedingung] Dem Benutzer werden die gewählten Einstellungen und die Schaltfläche zum Starten der Suche angezeigt
		\end{description}
		\item
		\begin{description}
			\item[Aktion] Der Benutzer startet die Suche (\ref{f:suche_starten}), bringt das Programm in den Hintergrund und wartet
			\item[Nachbedingung] Es ist ein Signalton zu hören (\ref{fw:signalton})
		\end{description}
		\item
		\begin{description}
			\item[Aktion] Der Benutzer bringt die GUI in den Vordergrund
			\item[Nachbedingung] Die Suche ist beendet; die Ergebnisse werden angezeigt (\ref{f:suchergebnisse_anzeigen})
		\end{description}
		\item
		\begin{description}
			\item[Aktion] Der Benutzer kehrt zur Bibliothek zurück
			\item[Nachbedingung] Die Bibliothek wird angezeigt (\ref{f:bibliothek_anzeigen})
		\end{description}

		\item
		\begin{description}
			\item[Aktion] Der Benutzer lässt sich die eben vorgenommenen Einstellungen anzeigen
			\item[Nachbedingung] Die GUI zeigt an, dass der Signalton eingeschaltet ist
		\end{description}
		\item
		\begin{description}
			\item[Aktion] Der Benutzer schaltet den Signalton aus (\ref{fw:signalton_einaus})
			\item[Nachbedingung] Das Programm ist beendet
		\end{description}
		\item
		\begin{description}
			\item[Aktion] Der Benutzer beendet das Programm (\ref{f:beenden})
			\item[Nachbedingung] Das Programm ist beendet
		\end{description}
		\item
		\begin{description}
			\item[Aktion] Der Benutzer startet das Programm (\ref{f:programmstart})
			\item[Nachbedingung] Das Programm zeigt die Bibliothek an
		\end{description}
		\item
		\begin{description}
			\item[Aktion] Der Benutzer lässt sich die eben vorgenommenen Einstellungen bezüglich des Signaltons anzeigen
			\item[Nachbedingung] Die GUI zeigt an, dass der Signalton ausgeschaltet ist
		\end{description}
		\item
		\begin{description}
			\item[Aktion] Der Benutzer nimmt alle Schritte vor um eine Suche zu starten und wartet
			\item[Nachbedingung] die Suche ist beendet und die Ergebnisse werden angezeigt; es wird kein Signalton abgespielt
		\end{description}
	\end{enumerate}

	\item Videos in Videoformaten abspielen
	\begin{enumerate}[leftmargin=0pt]
		\item
		\begin{description}
			\item[Vorbedingung] Das Programm zeigt die Übersicht eines Videodatensatzes (Videos in Videoformaten)
			\item[Aktion] Der Benutzer wählt ein Video aus (\ref{f:anzeigen_groessere_darstellung})
			\item[Nachbedingung] Das gewählte Video wird im Videoplayer angezeigt
		\end{description}
		\item
		\begin{description}
			\item[Aktion] Der Benutzer startet das Video (\ref{fw:echtes_video_abspielen})
			\item[Nachbedingung] Das gewählte Video wird im Videoplayer abgespielt
		\end{description}
		\item
		\begin{description}
			\item[Aktion] Der Benutzer rechtsklickt auf das Video und wählt ein Suchverfahren (\ref{f:auswahl_suchverfahren})
			\item[Nachbedingung] Die Parameterwahl wird angezeigt
		\end{description}
		\item
		\begin{description}
			\item[Aktion] Der Benutzer wählt die Parameter nach belieben (\ref{f:parameterwahl}) und startet die Suche (\ref{f:suche_starten})
			\item[Nachbedingung] Das gewählte Video wird angezeigt (nicht abgespielt); die vorgenommenen Einstellungen werden angezeigt (\ref{f:ueberpruefung})
		\end{description}
	\end{enumerate}

	\item Präsentationsmodus
	\begin{enumerate}[leftmargin=0pt]
		\item
		\begin{description}
			\item[Vorbedingung]Das Programm zeigt die Bibliothek; die GUI befindet sich nicht im Präsentationsmodus
			\item[Aktion] Der Benutzer startet den Präsentationsmodus über das Menü (\ref{fw:präsentation})ss
			\item[Nachbedingung] Die GUI wird als Vollbild angezeigt; die Option den Vollbildmodus zu beenden ist verfügbar
		\end{description}
		\item
		\begin{description}
			\item[Aktion] Der Benutzer beendet den Vollbildmodus
			\item[Nachbedingung] Die GUI wird wie zu Beginn des Testszenarios angezeigt
		\end{description}
	\end{enumerate}

	\item Zoomen und Scrollen
	\begin{enumerate}[leftmargin=0pt]
		\item
		\begin{description}
			\item[Vorbedingung] Das Programm zeigt die Übersicht eines Bilddatensatzes; ein Bild ist ausgewählt
			\item[Aktion] Der Benutzer rollt das Mausrad vorwärts während sich der Cursor über der größeren Ansicht befindet (\ref{fw:zoom_scroll})
			\item[Nachbedingung] Es wird ein kleinerer Ausschnitt des Bildes größer angezeigt
		\end{description}
		\item
		\begin{description}
			\item[Aktion] Der Benutzer hält die Rechte Maustaste und bewegt den Cursor im Bereich der größeren Ansicht (\ref{fw:zoom_scroll})
			\item[Nachbedingung] Das Bild folgt dem Cursor, wodurch andere Bereiche des Bildes sichtbar werden; falls der Rand des gesamten Bildes sichtbar ist verschiebt sich der sichtbare Bereich jedoch nicht über den Rand hinaus
		\end{description}
		\item
		\begin{description}
			\item[Aktion] Der Benutzer rollt das Mausrad rückwärts (\ref{fw:zoom_scroll})
			\item[Nachbedingung] Es wird wieder ein größerer Ausschnitt des Bildes kleiner angezeigt
		\end{description}
	\end{enumerate}

	\item Suchergebnisse größer anzeigen
	\begin{enumerate}[leftmargin=0pt]
		\item
		\begin{description}
			\item[Vorbedingung] Das Programm zeigt die Übersicht der Ergebnisse einer Suche an
			\item[Aktion] Der Benutzer wählt ein Ergebnis \ref{fw:groesseres_suchergebnis}
			\item[Nachbedingung] Das Ergebnis wird in einer größeren Ansicht angezeigt
		\end{description}
	\end{enumerate}
	Falls \ref{fw:groesseres_suchergebnis} implementiert ist, sind hier ggf. die Funktionen \ref{f:video_abspielen}, \ref{fw:echtes_video_abspielen} und \ref{fw:zoom_scroll} analog zu deren Testfällen zu überprüfen.

	\item Suche abbrechen
	\begin{enumerate}[leftmargin=0pt]
		\item
		\begin{description}
			\item[Vorbedingung] Es wurden alle Einstellungen einer bevorstehenden Suche vorgenommen; Es werden die vorgenommenen Einstellungen angezeigt; eine Schaltfläche zum Starten der Suche wird angezeigt
			\item[Aktion] Der Benutzer startet die Suche
			\item[Nachbedingung] Eine Animation zeigt die Aktivität des Suchverfahrens an (\ref{f:fortschrittsanimation})
		\end{description}
		\item
		\begin{description}
			\item[Aktion] Der Benutzer bricht die Suche ab \ref{fw:suche_abbrechen}
			\item[Nachbedingung] Es werden wieder die vorgenommenen Einstellungen angezeigt; eine Schaltfläche zum Starten der Suche wird angezeigt
		\end{description}
		\item
		\begin{description}
			\item[Aktion] Der Benutzer startet die Suche \ref{f:suche_starten}
			\item[Nachbedingung] Eine Animation zeigt die Aktivität des Suchverfahrens an (\ref{f:fortschrittsanimation})
		\end{description}
		\item
		\begin{description}
			\item[Aktion] Der Benutzer wartet
			\item[Nachbedingung] Die Ergebnisse der Suche werden angezeigt \ref{f:suchergebnisse_anzeigen}
		\end{description}
	\end{enumerate}

	\item genaueres Feedback
	\begin{enumerate}[leftmargin=0pt]
		\item
		\begin{description}
			\item[Vorbedingung] Das Programm zeigt die Übersicht der Ergebnisse einer Suche an; der Benutzer hat die Möglichkeit ein Feedback in Form eines Wertes von 0 bis 10 festzulegen
			\item[Aktion] Der Benutzer wählt ein Feedback für eines der Ergebnisse \ref{fw:zehner_feedback}
			\item[Nachbedingung] Das gegebene Feedback wird dargestellt
		\end{description}
	\end{enumerate}
	Falls \ref{fw:zehner_feedback} implementiert ist, ist hier ggf. die Funktionen \ref{fw:lesezeichen} analog zu ihren Testfällen zu überprüfen.
\end{enumerate}
\pagebreak
