% !TeX spellcheck = de_DE_frami
	Es gibt für jede funktionale Anforderung einen Testfall.

\begin{description}
	\item[\req{TF 10}] Datensatz zur Bibliothek hinzufügen 
	\begin{enumerate}
		% Vorgang zu generisch beschrieben; ToDo: bestimmte Datensätze und Verfahren zur konkretisierung spezifizieren und hier verwenden
		\begin{description}
			\item[Vorbedingung] Programm ist startbereit; es wurden weder Datensätze noch Suchverfahren entfernt oder manipuliert
			\item[Aktion] Benutzer startet Programm (\req{F 10})
			\item[Nachbedingung] Startbildschirm der GUI wird angezeigt; die Bibliothek wird angezeigt (\req{F 70}) und enthält mindestens einen Datensatz
		\end{description}
		\item
		\begin{description}
			\item[Aktion] Der Benutzer doppelklickt einen Datensatz (\req{F 72})
			\item[Nachbedingung] Es wird eine Übersicht des Datensatzes angezeigt (\req{F 80})
		\end{description}
		\item
		\begin{description}
			\item[Aktion] Der Benutzer klickt auf ein Bild aus der Übersicht (\req{F 90})
			\item[Nachbedingung] Das Bild wird groß dargestellt (\req{F 85})
		\end{description}
		\item
		\begin{description}
			\item[Aktion] Der Benutzer rechtsklickt auf einen Bereich des Bildes der nicht annotiert ist
			\item[Nachbedingung] Es wird ein Kontextmenü angezeigt in dem mindestens ein Suchverfahren zur Auswahl bereitsteht (\req{F 30})
		\end{description}
		\item
		\begin{description}
			\item[Aktion] Der Benutzer fährt mit der Maus über ein Suchverfahren
			\item[Nachbedingung] Es wird eine Beschreibung des Suchverfahrens angezeigt (\req{F 121})
		\end{description}
		\item
		\begin{description}
			\item[Aktion] Der Benutzer linksklickt auf eines der Suchverfahren (\req{F 120})
			\item[Nachbedingung] Das angezeigte Fenster enthält die Parameter des Suchverfahrens und entsprechende UI-Elemente sie festzulegen
		\end{description}
		\item
		\begin{description}
			\item[Aktion] Der Benutzer nimmt seine Eingaben vor und bestätigt sie (\req{F 125})
			\item[Nachbedingung] Es werden die vorgenommenen Einstellungen angezeigt (\req{F 128}); eine Schaltfläche zum Starten der Suche wird angezeigt
		\end{description}
		\item
		\begin{description}
			\item[Aktion] Der Benutzer startet die Suche (\req{F 130})
			\item[Nachbedingung] Eine Animation zeigt die Aktivität des Suchverfahrens an (\req{F 140})
		\end{description}
		\item
		\begin{description}
			\item[Aktion] Der Benutzer wartet eine entsprechende Zeit
			\item[Nachbedingung] Eine Übersicht über die Ergebnisse der Suche wird angezeigt (\req{F 160})
		\end{description}
		\item
		\begin{description}
			\item[Aktion] Der Benutzer klickt auf ein Ergebnis
			\item[Nachbedingung] Das Ergebnis(-bild) wird groß dargestellt (\req{F 170})
		\end{description}
	\end{enumerate}
	\item[\req{TF 20}] was soll hier sein ???
	\begin{enumerate}
		\item 
		\begin{description}
			\item[Vorbedingung] Programm ist startbereit; ein korrekter Datensatz liegt auf der Festplatte; % "korrekt" = ?
			\item[Aktion] Benutzer startet Programm (\req{F 10})
			\item[Nachbedingung] Startbildschirm der GUI wird angezeigt; die Bibliothek wird angezeigt (\req{F 70})
		\end{description}
		\item
		\begin{description}
			\item[Aktion] Der Benutzer klickt die Schaltfläche zum Hinzufügen eines neuen Datensatzes (\req{F 72})
			\item[Nachbedingung] Ein Dialog der die Auswahl eines Datensatzes von der Festplatte ermöglicht wird angezeigt
		\end{description}
		\item
		\begin{description}
			\item[Aktion] Der Benutzer wählt einen korrekten Datensatz von der Festplatte und bestätigt seine Wahl (\req{F 71})
			\item[Nachbedingung] der Dialog wird nicht mehr angezeigt; die Liste der gewählten Datensätze enthält den eben gewählten Datensatz
		\end{description}
	\end{enumerate}

	\item[\req{TF 30}] Sprachwahl
	\begin{enumerate}
		\item
		\begin{description}
			\item[Vorbedingung] Programm ist startbereit; Einstellungen wurden noch nicht geändert
			\item[Aktion] Benutzer startet Programm (\req{F 10})
			\item[Nachbedingung] deutsches GUI wird angezeigt; Menüpunkt 'Sprache' ist sichtbar, es sind die Sprachen Deutsch und Englisch wählbar; Deutsch ist als aktuell gewählte Sprache markiert
		\end{description}
		\item
		\begin{description}
			\item[Aktion] Der Benutzer wählt über den Menpunkt 'Sprache' die englische Sprache aus (\req{FW 10})
			\item[Nachbedingung] die GUI wird auf englisch dargestellt
		\end{description}
		\item
		\begin{description}
			\item[Aktion] Der Benutzer beendet das Programm über den üblichen Schließen-Button (\req{F 20})
			\item[Nachbedingung] die GUI wird ist nicht mehr sichtbar; der Prozess hat terminiert
		\end{description}
		\item
		\begin{description}
			\item[Aktion] Der Benutzer startet das Programm (\req{F 10})
			\item[Nachbedingung] GUI wird auf Englisch angezeigt (\req{FW 10}); unter dem Menüpunt 'Sprache' ist nun Englisch als aktuell gewählte Sprache markiert
		\end{description}
	\end{enumerate}

	\item[\req{TF 40}] weitere Menüpunkte
	\begin{enumerate}
		\item
		\begin{description}
			\item[Vorbedingung] das Menü ist verfügbar; deutsch ist die gewählte Sprache (falls \req{FW 10} implementiert)
			\item[Aktion] der Benutzer wählt über den Menüpunkt ganz Rechts "About" aus (\req{F 50})
			\item[Nachbedingung] ein Dialog mit Informationen zum Programm wird angezeigt
		\end{description}
		\item
		\begin{description}
			\item[Aktion] der Benutzer wählt über den Menüpunkt ganz Rechts "Hilfe" aus (\req{F 40})
			\item[Nachbedingung] ein Dialog mit Hinweisen zur Benutzung des Programms wird angezeigt
		\end{description}
	\end{enumerate}
\end{description}
