% !TeX spellcheck = de_DE_frami
Die folgenden Testszenarien beschreiben den Arbeitsablauf des Benutzers im Detail.	

\begin{enumerate} [label=\bfseries /TS \arabic*0/]
	\item Grundfunktionalität % Vorgang zu generisch beschrieben; ToDo: bestimmte Datensätze und Verfahren zur konkretisierung spezifizieren und hier verwenden
	\begin{enumerate}
		\item
		\begin{description}
			\item[Vorbedingung] Programm ist startbereit; es wurden weder Datensätze noch Suchverfahren entfernt oder manipuliert
			\item[Aktion] Benutzer startet Programm (\req{F 10})
			\item[Nachbedingung] die Startansicht der GUI wird angezeigt; die Bibliothek wird angezeigt (\req{F 70}) und enthält mindestens einen Datensatz
		\end{description}
		\item
		\begin{description}
			\item[Aktion] Der Benutzer doppelklickt einen Datensatz (\req{F 72})
			\item[Nachbedingung] Es wird eine Übersicht des Datensatzes angezeigt (\req{F 80})
		\end{description}
		\item
		\begin{description}
			\item[Aktion] Der Benutzer klickt auf ein Bild aus der Übersicht (\req{F 90})
			\item[Nachbedingung] Das Bild wird groß dargestellt (\req{F 85})
		\end{description}
		\item
		\begin{description}
			\item[Aktion] Der Benutzer rechtsklickt auf einen Bereich des Bildes der nicht annotiert ist
			\item[Nachbedingung] Es wird ein Kontextmenü angezeigt in dem mindestens ein Suchverfahren zur Auswahl bereitsteht (\req{F 30})
		\end{description}
		\item
		\begin{description}
			\item[Aktion] Der Benutzer fährt mit der Maus über ein Suchverfahren
			\item[Nachbedingung] Es wird eine Beschreibung des Suchverfahrens angezeigt (\req{F 121})
		\end{description}
		\item
		\begin{description}
			\item[Aktion] Der Benutzer linksklickt auf eines der Suchverfahren (\req{F 120})
			\item[Nachbedingung] Das angezeigte Fenster enthält die Parameter des Suchverfahrens und entsprechende UI-Elemente sie festzulegen
		\end{description}
		\item
		\begin{description}
			\item[Aktion] Der Benutzer nimmt seine Eingaben vor und bestätigt sie (\req{F 125})
			\item[Nachbedingung] Es werden die vorgenommenen Einstellungen angezeigt (\req{F 128}); eine Schaltfläche zum Starten der Suche wird angezeigt
		\end{description}
		\item
		\begin{description}
			\item[Aktion] Der Benutzer startet die Suche (\req{F 130})
			\item[Nachbedingung] Eine Animation zeigt die Aktivität des Suchverfahrens an (\req{F 140})
		\end{description}
		\item
		\begin{description}
			\item[Aktion] Der Benutzer wartet
			\item[Nachbedingung] Eine Übersicht über die Ergebnisse der Suche wird angezeigt (\req{F 160})
		\end{description}
		\item
		\begin{description}
			\item[Aktion] Der Benutzer linksklickt auf ein Ergebnis
			\item[Nachbedingung] Das Ergebnis wird groß dargestellt (\req{F 170})
		\end{description}
		\item
		\begin{description}
			\item[Aktion] Der Benutzer wählt die Schaltfläche zur Rückkehr
			\item[Nachbedingung] Es werden wieder die vorgenommenen Einstellungen angezeigt (\req{F 128}); eine Schaltfläche zum Starten der Suche wird angezeigt
		\end{description}
		\item
		\begin{description}
			\item[Aktion] Der Benutzer Startet die Suche erneut
			\item[Nachbedingung] Eine Animation zeigt die Aktivität des Suchverfahrens an (\req{F 140})
		\end{description}
		\item
		\begin{description}
			\item[Aktion] Der Benutzer bricht die Suche ab
			\item[Nachbedingung] Es werden wieder die vorgenommenen Einstellungen angezeigt (\req{F 128}); eine Schaltfläche zum Starten der Suche wird angezeigt
		\end{description}
		\item
		\begin{description}
			\item[Aktion] Der Benutzer beendet das Programm über den üblichen Schließen-Button (\req{F 20})
			\item[Nachbedingung] die GUI wird ist nicht mehr sichtbar; der Prozess hat terminiert
		\end{description}
	\end{enumerate}

	\item Feedback und Lesezeichen
	\begin{enumerate}
		\item
		\begin{description}
			\item[Vorbedingung] Es wurde gerade eine Suche durchgeführt; es werden die Ergebnisse der Suche angezeigt
			\item[Aktion] Der Benutzer stellt ein positives und ein negatives Feedback für je ein Suchergebnis ein (\req{FW 190})
			\item[Nachbedingung] Das gegebene Feedback wird in der Übersicht dargestellt
		\end{description}
		\item
		\begin{description}
			\item[Aktion] Der Benutzer klickt auf die Schaltfläche zum Speichern der Suchergebnisse (\req{F 205}), gibt einen einen Namen für das Lesezeichen ein und beendet anschließend das Programm (\req{F 20})
			\item[Nachbedingung] Das Programm wurde beendet
		\end{description}
		\item
		\begin{description}
			\item[Aktion] Der Benutzer startet das Programm erneut (\req{FW 10})
			\item[Nachbedingung] Die Startansicht der GUI wird angezeigt
		\end{description}
		\item
		\begin{description}
			\item[Aktion] Der Benutzer klickt auf die Schaltfläche für das Anzeigen der Lesezeichen (\req{F 61})
			\item[Nachbedingung] in der angezeigten Liste ist mindestens das zuvor angelegte Lesezeichen zu finden
		\end{description}
		\item
		\begin{description}
			\item[Aktion] Der Benutzer wählt das zuvor angelegte Lesezeichen (\req{F 60}, \req{F 205})
			\item[Nachbedingung] Die bereits zuvor angezeigten Suchergebnisse werden in der selben Reihenfolge wieder angezeigt; das zuvor festgelegte Feedback wird wieder angezeigt
		\end{description}
	\end{enumerate}

	\item erweiterte Funktionen zu Bilddatensätzen
	\begin{enumerate}
		\item
		\begin{description}
			\item[Vorbedingung] Es wurde ein Datensatz (bestehend aus Bildern) gewählt; das Programm zeigt die Übersicht der Bilder des Datensatzes; es ist ein Foto eines realen Motivs im Datensatz vorhanden; das Foto weist annotierte Bildbereiche auf
			\item[Aktion] Der Benutzer wählt dieses Bild zur vergrößerten Ansicht (\req{F 85})
			\item[Nachbedingung] das Bild wird in vergrößerter Ansicht angezeigt; es werden die annotierten Bildbereiche angezeigt (\req{F 100})
		\end{description}
		\item
		\begin{description}
			\item[Aktion] Der Benutzer verwendet das Mausrad zum Zoomen und Drag-and-Drop zum Scrollen in der vergrößerten Ansicht (\req{F 86})
			\item[Nachbedingung] Die vergröerte Ansicht zoomt und scrollt
		\end{description}
		\item
		\begin{description}
			\item[Aktion] Der Benutzer rechtsklickt auf einen annotierten Bildbereich (\req{F 105})
			\item[Nachbedingung] Es wird ein Kontextmenü angezeigt in dem mindestens ein Suchverfahren zur Auswahl bereitsteht (\req{F 30})
		\end{description}
		\item
		\begin{description}
			\item[Aktion] Der Benutzer linksklickt auf ein Suchverfahren (\req{F 120})
			\item[Nachbedingung] Es wird der gewählte annotierte Bildbereich angezeigt (\req{F 128}); eine Schaltfläche zur Rückkehr zur Übersicht des Datensatzes wird angezeigt
		\end{description}
		\item
		\begin{description}
			\item[Aktion] Der Benutzer linksklickt die Schaltfläche zur Rückkehr (\req{F 129})
			\item[Nachbedingung] Das eben gewählte Bild wird in der vergrößerten Ansicht dargestellt
		\end{description}
		\item
		\begin{description}
			\item[Aktion] Der Benutzer wählt das Werkzeug zur Auswahl eines Bildbereiches und wählt per Drag-and-Drop in der vergrößerten Ansicht einen Bildbereich (\req{F 110})
			\item[Nachbedingung] Der gewählte Bereich wird dargestellt
		\end{description}
		\item
		\begin{description}
			\item[Aktion] Der Benutzer rechtsklickt in den gewählten Bereich und linksklickt auf ein Suchverfahren (\req{F 120})
			\item[Nachbedingung] Es wird der gewählte Bildbereich angezeigt (\req{F 128}); eine Schaltfläche zur Rückkehr zur Übersicht des Datensatzes wird angezeigt
		\end{description}
		\item
		\begin{description}
			\item[Aktion] Der Benutzer linksklickt die Schaltfläche zur Rückkehr (\req{F 129})
			\item[Nachbedingung] Das eben gewählte Bild wird in der vergrößerten Ansicht dargestellt; der gewählte Bildbereich wird angezeigt
		\end{description}
		\item
		\begin{description}
			\item[Aktion] Der Benutzer verwendet das Mausrad zum Zoomen und Drag-and-Drop zum Scrollen in der vergrößerten Ansicht (\req{F 86})
			\item[Nachbedingung] Die vergröerte Ansicht zoomt und scrollt
		\end{description}
	\end{enumerate}

	\item Datensatz zur Bibliothek hinzufügen
	\begin{enumerate}
		\item
		\begin{description}
			\item[Vorbedingung] Programm ist startbereit; ein korrekter Datensatz liegt auf der Festplatte; % "korrekt" = ?
			\item[Aktion] Benutzer startet Programm (\req{F 10})
			\item[Nachbedingung] Startbildschirm der GUI wird angezeigt; die Bibliothek wird angezeigt (\req{F 70})
		\end{description}
		\item
		\begin{description}
			\item[Aktion] Der Benutzer klickt die Schaltfläche zum Hinzufügen eines neuen Datensatzes (\req{F 72})
			\item[Nachbedingung] Ein Dialog der die Auswahl eines Datensatzes von der Festplatte ermöglicht wird angezeigt
		\end{description}
		\item
		\begin{description}
			\item[Aktion] Der Benutzer wählt einen korrekten Datensatz von der Festplatte und bestätigt seine Wahl (\req{F 71})
			\item[Nachbedingung] der Dialog wird nicht mehr angezeigt; die Liste der gewählten Datensätze enthält den eben gewählten Datensatz
		\end{description}
	\end{enumerate}

	\item Sprachwahl
	\begin{enumerate}
		\item
		\begin{description}
			\item[Vorbedingung] Programm ist startbereit; Einstellungen wurden noch nicht geändert
			\item[Aktion] Benutzer startet Programm (\req{F 10})
			\item[Nachbedingung] deutsches GUI wird angezeigt; Menüpunkt 'Sprache' ist sichtbar, es sind die Sprachen Deutsch und Englisch wählbar; Deutsch ist als aktuell gewählte Sprache markiert
		\end{description}
		\item
		\begin{description}
			\item[Aktion] Der Benutzer wählt über den Menpunkt 'Sprache' die englische Sprache aus (\req{FW 10})
			\item[Nachbedingung] die GUI wird auf englisch dargestellt
		\end{description}
		\item
		\begin{description}
			\item[Aktion] Der Benutzer beendet das Programm über den üblichen Schließen-Button (\req{F 20})
			\item[Nachbedingung] die GUI wird ist nicht mehr sichtbar; der Prozess hat terminiert
		\end{description}
		\item
		\begin{description}
			\item[Aktion] Der Benutzer startet das Programm (\req{F 10})
			\item[Nachbedingung] GUI wird auf Englisch angezeigt (\req{FW 10}); unter dem Menüpunt 'Sprache' ist nun Englisch als aktuell gewählte Sprache markiert
		\end{description}
	\end{enumerate}

	\item weitere Menüpunkte
	\begin{enumerate}
		\item
		\begin{description}
			\item[Vorbedingung] das Menü ist verfügbar; deutsch ist die gewählte Sprache (falls \req{FW 10} implementiert)
			\item[Aktion] der Benutzer wählt über den Menüpunkt ganz Rechts 'About' aus (\req{F 50})
			\item[Nachbedingung] ein Dialog mit Informationen zum Programm wird angezeigt
		\end{description}
		\item
		\begin{description}
			\item[Aktion] der Benutzer wählt über den Menüpunkt ganz Rechts 'Hilfe' aus (\req{F 40})
			\item[Nachbedingung] ein Dialog mit Hinweisen zur Benutzung des Programms wird angezeigt
		\end{description}
	\end{enumerate}
\end{enumerate}
