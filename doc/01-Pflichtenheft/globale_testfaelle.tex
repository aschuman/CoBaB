% !TeX spellcheck = de_DE_frami
	Es gibt für jede funktionale Anforderung einen Testfall.
\begin{enumerate} [label=\bfseries /TF \arabic*0/]
	\item Das Programm starten /F 10/
	\item Das Programm beenden /F 20/
	\item Eine Sprache auswählen /F 30/
	\item Ein Bild aus der Medienbibliothek auswählen /F 40/, /F 50/, /F 60/
	\item Ein Video aus der Medienbibliothek auswählen /F 60/
	\item Eine Annotation auf die ausgewählte Datei machen /F 70/
	\item Einen Suchverfahren auswählen /F 80/
	\item (W) Mehrere Suchverfahren wählen /F 260/
	\item Die Suche anfordern /F 90/
	\item Die Suche terminieren /F 100/
	\item (W) Die Suche pausieren /F 110/
	\item (W) Die pausierte Suche fortsetzen /F 120/
	\item (W) Den Zurück-Button anfordern /F 130/
	\item (W) Die Suchergebnisse entfernen /F 140/
	\item (W) Benachrichtigungston ein- und ausschalten /F 150/, /F 160/
	\item Einen Suchergebnis auswählen und anzeigen /F 170/, /F 180/
	\item (W) Ein ganzes Video abspielen /F 190/
	\item Zutreffende Suchergebnisse auswählen /F 200/
	\item Neue Suche mit den ausgewählten Suchergebnissen starten /F 210/
	\item (W) Speichern der Suchergebnisse /F 220/, /F 230/
	\item (W) Aufrufen von gespeicherten Suchergebnissen /F 240/, /F 250/
	\item Aufrufen des Help-Menüs /F 270/
	\item Aufrufen des About-Menüs /F 280/
\end{enumerate}