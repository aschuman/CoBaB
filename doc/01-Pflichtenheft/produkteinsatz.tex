\subsection{Andwendungsbereiche}

Das Programm kann in viele unterschiedlichen Bereichen genutzt werden. Es kann dabei helfen, Fotoalben und Datenbanken nach Personen, Orten und Fahrzeugen zu durchsuchen. Bei Videoaufzeichnungen öffentlicher Veranstaltungen kann nach Firmenlogos gesucht werden. Die zugrundeliegende Videoaufzeichnung sollte als Folge von Einzelbildern vorliegen.

\subsection{Zielgruppen}

Den Wissenschaftlern des Fraunhofer Instituts wird es möglich, vorhandene Datensätze mit Hilfe ihrer inhaltsbasierten Suchalgorithmen zu durchsuchen. Das so entstandene Endprodukt ist dann marktreif.
Eine mögliche Zielgruppe ist der private Nutzer, der aus seinen Fotoalben ähnliche Fotos heraussuchen möchte. 
Außerdem können Unternehmen als weitere Zielgruppe angesprochen werden: Für sie ist es möglich, die Effizienz ihrer Werbung durch das Heraussuchen ihres Logos zu prüfen. 
Des weiteren können Polizei oder private Sicherheitsfirmen Überwachungsvideos nach Fahrzeugen oder Personen durchsuchen.

\subsection{Betriebsbedingungen}

Es müssen folgende Bedingungen erfüllt werden:

\begin{itemize}

\item Ubuntu 14 oder Windows 8 als Betriebssystem.
\item farbiges Display, auf dem Bilder und Videos laufen können 

\end{itemize}
