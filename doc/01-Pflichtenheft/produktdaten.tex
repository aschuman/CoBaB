\begin{itemize}
	\item Bilder und Videos bilden den Input des Programms.
	\item Die Annotationen eines Bildes sind in der zu diesem Bild gehörenden Datei zu speichern, die Konkretisierung erfolgt erst 		in der Entwurfsphase.
	\item Die Einstellungsdatei speichert die vom Benutzer gewählte Sprache.
	\item Es gibt eine Sprachdatei, die die Übersetzung vom Deutschen ins Englische und umgekehrt ermöglicht.
	\item Es wird eine Suchhistorie generiert, die letzten Suchergebnisse sind somit abrufbar. 
	\item Die zuletzt verwendeten Datensätze werden gespeichert.
	\item Lesezeichen verweisen auf spezielle Datensätze.
	\item Die Beschreibung des Ziels und der Aufgaben des Browsers ist zu speichern.
	\item Das Programm greift auf Suchverfahren und dynamisch gebundene Bibliotheken zu.
	\item Falls das /WK 110/ umgesetzt wird, sollen die Vorschaubildchen gespeichert werden.
\end{itemize}
\pagebreak