\begin{itemize}
	\item Bilder und Videos inklusive Annotationen bilden den Input des Programms.
	\item Die Einstellungsdatei speichert die vom Benutzer gewählte Sprache, falls \ref{fw:sprache} umgesetzt wird. Falls \ref{fw:signalton} umgesetzt wird, muss auch die Einstellung zum Signalton in der Einstellungsdatei gespeichert werden.
	\item Falls \ref{fw:sprache} umgesetzt wird, müssen die Übersetzungen gespeichert werden.
	\item Es wird eine Suchhistorie gespeichert, die letzten Suchergebnisse sind somit über mehrere Programmstarts abrufbar, falls \ref{fw:chronik} umgesetzt wird. 
	\item Eine Historie der zuletzt verwendeten Datensätze wird gespeichert, falls \ref{fw:speichern_historie} umgesetzt wird.
	\item Lesezeichen, die durch den Benutzer erstellt werden, speichern spezielle Suchergebnisse, falls \ref{fw:lesezeichen} umgesetzt wird.
	\item Die Beschreibung des Ziels und der Aufgaben des Browsers ist für das Hilfe-Menü zu speichern.
	\item Das Programm greift auf Suchverfahren und dynamisch gebundene Bibliotheken zu.
	\item Falls \ref{fw:vorschaubild} umgesetzt wird, sollen die Vorschaubildchen gespeichert werden.
\end{itemize}
\pagebreak
