\begin{itemize}
\item Datensatz\newline
Die Bilder und Videos bilden eine Grundlage der Anwendung, außerdem werden auch die Vorannotationen gespeichert.
\item Datensatzschnittstelle\newline
Die Datensatzschnittstelle bietet sowohl der Anwendung als auch den Suchverfahren einen einheitlichen Zugriff auf die hinterlegten Daten. Diese Schnittstelle wird von der Datensatzkomponente implementiert.
\item Suchschnittstelle\newline
Die Suchschnittstelle ermöglicht eine einheitliche Verwendung der Suchalgorithmen.
\end{itemize}

\subsection{Model}
\begin{itemize}
\item Konfigurationsdaten\newline
Die Konfigurationsdaten passen die Anwendung an die Präferenzen des Benutzers an: Die gewählte Sprache und die Aktivierung bzw. Deaktivierung des Benachrichtigungstons werden gespeichert und beim erneuten Starten der Anwendung als Voreinstellung gesetzt. Außerdem wird eine Hilfe-Datei gespeichert.
\item Datensatzkomponente\newline
Die Datensatzkomponente implementiert die Datensatzschnittstelle, um auf die Bilder bzw. Videos zuzugreifen.
\item Suchanfrage\newline
Die Anwendung generiert eine Suchanfrage, die aus gewähltem Suchmuster, gewünschtem Algorithmus, den Suchparametern und dem Suchraum besteht. Der Suchraum legt den Bereich fest, in dem gesucht werden soll.
\item Suchchronik\newline
Der Benutzer kann sich die letzten Suchergebnisse anzeigen lassen und außerdem Lesezeichen für spezielle Ergebnisse speichern.
\end{itemize}

\subsection{View}
\begin{itemize}
\item Mainwindow \newline
Das Mainwindow ist beim Starten des Programms zu sehen. Anfangs ist nur eine randomisierte Auswahl an Datensätzen zu sehen und es wird jedes mal beim erneuten Öffnen des Programms ein Datensatzview aus den meist genutzten Datensätzen generiert. Man kann entweder einen Datensatz von den Angezeigten wählen oder man öffnet eine Datensatzauswahl-Dialogfenster und man kann sich einen neuen wählen.
\item Menü-Anzeige \newline
Das Menü beinhaltet Bequemlichkeitsfunktionen wie \textbf{Datei}, wo zusätzliche Das Beenden des Programms und die Datensatzauswahl ermöglicht ist, \textbf{Sprachauswahl}, \textbf{Hilfe}, wo es Informationen über das Programm gibt oder Anweisungen zum Benutzen der Anwendung, \textbf{Chronik} mit gespeicherten früheren Suchen, \textbf{Lesezeichen}, die der Benutzer selbst aus seinen Suchergebnissen herstellt.
\item Foto-Anzeige \newline
Die Foto-Anzeige erlaubt das Browsen durch den Fotos im gewählten Datensatz mit bekannten Funktionen wie "voriges", "nächstes", Zoom und Vollbildmodus. Es gibt zusätzlich eine Option für das Wählen eines bestimmten Bereichs des Bildes. Durch einen Rechtsklick bestimmt man den Suchalgorithmus, der auf das Foto verwendet wird.
\item Videoplayer \newline
Das Videoplayer ist ähnlich der Foto-Anzeige aufgebaut, außer den üblichen Funktionen zum Abspielen und Pause.
\item Datensatzauswahl \newline
Die Datensatzauswahl stellt ein Dialogfenster dar, wo ein Datensatz aus allen Möglichen ausgewählt wird.
\end{itemize}	

\pagebreak
