Wer hat nicht schon einmal das Internet nach Bildern von sich oder Freunden durchsucht? Dabei wird häufig textbasierte Suche verwendet. Man tippt also einen Namen ein und erhält alle Bilder, auf denen die Person sein könnte, weil der Name der gesuchten Person im Text nebenan erwähnt wird. 
\newline
In Zukunft wird allerdings die inhaltsbasierte Suche immer wichtiger. Die Menge an verfügbaren Daten nimmt stetig zu und es treten Probleme mit der Datenhandhabung auf. Ohne inhaltsbasierte Suche ist eine Navigation durch die Datenbestände sehr aufwändig.
Nach einem Urlaub in London beispielsweise hat man sehr viele Urlaubsfotos. Sieht man die Fotos durch und entdeckt eine Aufnahme vom London Eye und möchte sich alle weiteren Fotos mit diesem Motiv anzeigen lassen, dann ist die inhaltsbasierte Suche hilfreich.
\newline
Unser Projekt soll dies nun basierend auf Datensätzen der Anwender und mit den Suchverfahren des Fraunhofer IOSB ermöglichen.
\newline
Dieses Pflichtenheft soll dabei einen ersten Einblick in die Anwendung liefern, indem es u.a. die Muss- und Wunschkriterien, die Produktanforderungen und die Benutzerschnittstelle erläutert.
\pagebreak